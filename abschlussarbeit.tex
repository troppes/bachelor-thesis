%%%%%%%%%%%%%%%%%%% vorlage.tex %%%%%%%%%%%%%%%%%%%%%%%%%%%%%
%
% LaTeX-Vorlage zur Erstellung von Projekt-Dokumentationen 
% im Fachbereich Informatik der Hochschule Trier
%
% Basis: Vorlage svmono des Springer Verlags
%
%%%%%%%%%%%%%%%%%%%%%%%%%%%%%%%%%%%%%%%%%%%%%%%%%%%%%%%%%%%%%
\documentclass[envcountsame,envcountchap, deutsch]{i-studis}

\usepackage{makeidx}         	% Index
\usepackage{multicol}        	% Zweispaltiger Index
%\usepackage[bottom]{footmisc}	% Erzeugung von Fu�noten

%%-----------------------------------------------------
%\newif\ifpdf
%\ifx\pdfoutput\undefined
%\pdffalse
%\else
%\pdfoutput=1
%\pdftrue
%\fi
%%--------------------------------------------------------
%\ifpdf
\usepackage[pdftex]{graphicx}
\usepackage{epstopdf}
\usepackage[pdftex,plainpages=false]{hyperref}
%\else
%\usepackage{graphicx}
%\usepackage[plainpages=false]{hyperref}
%\fi

%%-----------------------------------------------------
\usepackage{color}				% Farbverwaltung
%\usepackage{ngerman} 			% Neue deutsche Rechtsschreibung
\usepackage[english, ngerman]{babel}
%\usepackage[latin1]{inputenc} 	% Ermöglicht Umlaute-Darstellung
\usepackage[utf8]{inputenc}  	% Ermöglicht Umlaute-Darstellung unter Linux (je nach verwendetem Format)

%-----------------------------------------------------
\usepackage{listings} 			% Code-Darstellung
\lstset
{
	basicstyle=\scriptsize, 	% print whole listing small
	keywordstyle=\color{blue}\bfseries,
								% underlined bold black keywords
	identifierstyle=, 			% nothing happens
	commentstyle=\color{red}, 	% white comments
	stringstyle=\ttfamily, 		% typewriter type for strings
	showstringspaces=false, 	% no special string spaces
	framexleftmargin=7mm, 
	tabsize=3,
	showtabs=false,
	frame=single, 
	rulesepcolor=\color{blue},
	numbers=left,
	linewidth=146mm,
	xleftmargin=8mm
}
\usepackage{textcomp} 			% Celsius-Darstellung
\usepackage{amssymb,amsfonts,amstext,amsmath}	% Mathematische Symbole
\usepackage[german, ruled, vlined]{algorithm2e}
\usepackage[a4paper]{geometry} % Andere Formatierung
\usepackage{bibgerm}
\usepackage{array}
\usepackage{xcolor}
\hyphenation{ Ele-men-tar-ob-jek-te  ab-ge-tas-tet Aus-wer-tung House-holder-Matrix Le-ast-Squa-res-Al-go-ri-th-men} 		% Weitere Silbentrennung bei Bedarf angeben
\setlength{\textheight}{1.1\textheight}
\pagestyle{myheadings} 			% Erzeugt selbstdefinierte Kopfzeile
\makeindex 						% Index-Erstellung

\lstdefinelanguage{JSON}{
    numbers=left,
    numberstyle=\scriptsize,
    stepnumber=1,
    numbersep=8pt,
    showstringspaces=false,
    breaklines=true,
    frame=lines,
    literate=
     *{0}{{{\color{blue}0}}}{1}
      {1}{{{\color{blue}1}}}{1}
      {2}{{{\color{blue}2}}}{1}
      {3}{{{\color{blue}3}}}{1}
      {4}{{{\color{blue}4}}}{1}
      {5}{{{\color{blue}5}}}{1}
      {6}{{{\color{blue}6}}}{1}
      {7}{{{\color{blue}7}}}{1}
      {8}{{{\color{blue}8}}}{1}
      {9}{{{\color{blue}9}}}{1}
      {:}{{{\color{red}{:}}}}{1}
      {,}{{{\color{red}{,}}}}{1}
      {\{}{{{\color{blue}{\{}}}}{1}
      {\}}{{{\color{blue}{\}}}}}{1}
      {[}{{{\color{blue}{[}}}}{1}
      {]}{{{\color{blue}{]}}}}{1},
}

\newcommand\YAMLcolonstyle{\color{red}}
\newcommand\YAMLkeystyle{\color{black}}
\newcommand\YAMLvaluestyle{\color{blue}}
\newcommand\ProcessThreeDashes{\llap{\color{cyan}\mdseries-{-}-}}

\lstdefinelanguage{YAML}{
  keywords={true,false,null,y,n},
  keywordstyle=\color{blue},
  comment=[l]{\#},
  morecomment=[s]{/*}{*/},
  commentstyle=\color{red}
  moredelim=[l][\color{orange}]{\&},
  moredelim=[l][\color{magenta}]{*},
  moredelim=**[il][\YAMLcolonstyle{:}\YAMLvaluestyle]{:},   % switch to value style at :
  morestring=[b]',
  morestring=[b]",
  literate =    {---}{{\ProcessThreeDashes}}3
                {>}{{\textcolor{red}\textgreater}}1     
                {|}{{\textcolor{red}\textbar}}1 
                {\ -\ }{{\mdseries\ -\ }}3,
}

%--------------------------------------------------------------------------
\begin{document}
%------------------------- Titelblatt -------------------------------------
\title{Evaluation und prototypische Implementierung einer Enterprise-Suchmaschine in das Dietrich Online Projekt}
\subtitle{Evaluation and prototypical implementation of an enterprise search engine in the Dietrich Online Project}
%---- Die Art der Dokumentation kann hier ausgewählt werden---------------
%\project{Bachelor-Projektarbeit}
\project{Bachelor-Abschlussarbeit}
%\project{Master-Projektstudium}
%\project{Master-Abschlussarbeit}
%\project{Seminar zur Vorlesung ...}
%\project{Hausarbeit zur Vorlesung ...}
%--------------------------------------------------------------------------
\supervisor{Prof. Dr. Christoph Schmitz} 		% Betreuer der Arbeit
\author{ Florian Reitz } 							% Autor der Arbeit
\address{Trier,} 							% Im Zusammenhang mit dem Datum wird hinter dem Ort ein Komma angegeben
\submitdate{28.3.2020} 				% Abgabedatum
%\begingroup
%  \renewcommand{\thepage}{title}
%  \mytitlepage
%  \newpage
%\endgroup
\begingroup
  \renewcommand{\thepage}{Titel}
  \mytitlepage
  \newpage
\endgroup
%--------------------------------------------------------------------------
\frontmatter 
%--------------------------------------------------------------------------
\preface

Diese Arbeit entstand als Abschlussarbeit an der Hochschule Trier in Zusammenarbeit mit der Bibliothek der Universität Trier. 
\newline
\newline
Die Idee zu dieser Arbeit entwickelte sich während meiner Arbeit am Dietrich online Projekt. Ich möchte diese Stelle nutzen, um mich beim Dietrich online Team und vor allem bei Herrn Kock für die Unterstützung zu bedanken.
\newline
Join
Query
\newline
Ein besonderer Dank gilt auch Herrn Professor Schmitz und Herrn Röpke für die Betreuung dieser Arbeit.
\newline
\newline
In dieser Arbeit wird aus Gründen der besseren Lesbarkeit das generische Maskulinum verwendet. Die dabei gewählte Form bezieht sich auf alle Geschlechter des Spektrums. 
\newline
Zudem wird DietrichOnline anstelle Dietrich online für eine bessere Lesbarkeit verwendet.
\newline
\newline
Der Code, der für diese Arbeit erstellt wurde, ist unter: \url{https://seafile.rlp.net/f/69909eeffcbd406abe00/} zu finden. Das Passwort für den Download ist: Bachelor2020.
\newline
\newline
Trier, 2020
\newline
\noindent Florian Reitz				% Vorwort (optional)
\kurzfassung

%% deutsch
\paragraph*{German}
Diese Arbeit handelt von der Analyse diverser Enterprise-Suchmaschinen für das DietrichOnline-Projekt \ref{dietrichonline}. Dabei wurden die Suchmaschinen nach einer Anforderungsliste untersucht und die verbleibenden Kandidaten für einen Ersteindruck aufgesetzt. 

Nachdem sich für Elasticsearch entschieden wurde, wurde diese in einer Docker-Umgebung aufgesetzt. Dabei wurde auf eine verschlüsselte Kommunikation zwischen den einzelnen Systemen viel Wert gelegt.

Im letzten Teil der Arbeit wurde zudem eine prototypische Implementierung in das Dietrich Online Projekt vorgenommen. Dafür wurde die Suche, sowie die Auto-Vervollständigung auf die Suchmaschine umgezogen.

%% englisch
\paragraph*{English}

This thesis analyzes various enterprise search-engines for the Dietrich Online project \ref{dietrichonline}. The search-engines were checked via a feature list and four of the remaining search engines were set up for a first impression.

After the decision was made for Elasticsearch, it was set up in a Docker environment. Great importance was attached to encrypted communication between the individual systems.

The last part of this thesis is a prototype implementation of the search engine in the Dietrich-Online project. The search and the auto-completion function were set up to use Elasticsearch. 			% Kurzfassung Deutsch/English
\tableofcontents 						% Inhaltsverzeichnis
\listoffigures 							% Abbildungsverzeichnis (optional)
\listoftables 							% Tabellenverzeichnis (optional)
%--------------------------------------------------------------------------
\mainmatter                        		% Hauptteil (ab hier arab. Seitenzahlen)
%--------------------------------------------------------------------------
% Die Kapitel werden in separaten .tex-Dateien abgelegt und hier eingebunden.
\chapter{Einleitung und Problemstellung}


Die Suche des DietrichOnline-Projektes \ref{dietrichonline} arbeitet aktuell auf einer MariaDB Datenbank. In dieser werden bei jeder Suchanfrage diverse Tabellen mithilfe von SQL-Joins zusammengebaut und daraufhin dem Nutzer ausgegeben. Bei den Datenmengen, welche sich aktuell in der Datenbank befinden, circa 1.4 Millionen Einträge, werden Ladezeiten unangenehm lang. Daher wurden die maximale Anzahl von Suchergebnissen, welche ein Nutzer aktuell bekommen kann, auf 1001 begrenzt. 

Damit die Nutzer ein möglichst gutes Sucherlebnis haben, sollen sogenannte Enterprise-Suchmaschinen evaluiert werden. Diese indexieren die Daten in einer Weise, welche es ermöglicht, viele Datensätze schnell zu durchsuchen. 

Im ersten Schritt werden nun diverse Suchmaschinen nach einer Kriterienliste analysiert. Im zweiten Schritt werden die vier am besten passenden Suchmaschinen daraufhin für einen Ersteindruck aufgesetzt.

Sobald ein Kandidat ausgewählt ist, wird dieser, wenn möglich in einer Docker-Umgebung aufgesetzt. Dabei werden auch die benötigten Datensätze indexiert.

Zuletzt wird eine prototypische Implementierung in das DietrichOnline-Projekt vorgenommen. Hierbei wird die aktuelle Suche durch die Enterprise-Suchmaschine ersetzt und um einige Funktionen, wie eine bessere Schnellsuche, erweitert. 
%\chapter{LaTeX-Bausteine}\label{Stile}

Der Text wird in bis zu drei Ebenen gegliedert:

\begin{enumerate}
  \item Kapitel ( \verb \chapter{Kapitel} ), \index{Kapitel}
  \item Unterkapitel  ( \verb \section{Abschnitt} ) und
  \item Unterunterkapitel  ( \verb \subsection{Unterabschnitte} ).
\end{enumerate}

\section{Abschnitt}\index{Abschnitt}
Text der Gliederungsebene 2.


\subsection{Unterabschnitt} \index{Unterabschnitt}
Text der Gliederungsebene 3.
Text Text Text Text Text Text Text Text Text Text Text Text Text Text Text
Beispiel f�r Quelltext\index{Quelltext} \\[2 ex]
\noindent
\begin{minipage}{1.0\textwidth} \small
\begin{lstlisting}
	Prozess 1:
	
	Acquire();
		a := 1;
	Release();
	...
	Acquire();
	if(b == 0)
	{					
		c := 3;
		d := a;
	}				
	Release();
\end{lstlisting}
\end{minipage}

\vspace{2cm}
\noindent
\begin{minipage}{1.0\textwidth} \small
\begin{lstlisting}
	Prozess 2:
	
	Acquire();
		b := 1;
	Release();
	...
	Acquire();
	if(a == 0)
	{					
		c := 5;
		d := b;
	}				
	Release();
\end{lstlisting}
\end{minipage}
\vskip 1em

Gr��ere Code-Fragmente sollten im Anhang eingef�gt werden.

\section{Abbildungen und Tabellen}

Abbildung\index{Abbildung} und Tabellen\index{Tabelle} werden zentriert eingef�gt. Grunds�tzlich sollen sie
erst dann erscheinen, nach dem sie im Text angesprochen wurden (siehe Abb. \ref{a1}). Abbildungen und Tabellen (siehe Tabelle \ref{t1}) k�nnen
im (flie�enden) Text (\verb here ), am Seitenanfang (\verb top ), am Seitenende
(\verb bottom ) oder auch gesammelt auf einer nachfolgenden Seite (\verb page )
oder auch ganz am Ende der Ausarbeitung erscheinen. Letzteres sollte man nur
dann w�hlen, wenn die Bilder g�nstig zusammen zu betrachten sind und die
Ausarbeitung nicht zu lang ist ($< 20$ Seiten).

\begin{figure} %[hbtp]
	\centering
		\includegraphics{images/p1ReadSeq.pdf}
	\caption{Bezeichnung der Abbildung}
	\label{a1}
\end{figure}

\begin{table} %[hbtp]
	\centering
		\begin{tabular}{l | l l l l}
		\textbf{Prozesse} & \textbf{Zeit} $\rightarrow$ \\
		\hline
			$P_{1}$ & $W(x)1$ \\
			$P_{2}$ & & $W(x)2$ \\
			$P_{3}$ & & $R(x)2$ & & $R(x)1$\\
			$P_{4}$ & & & $R(x)2$ & $R(x)1$\\
		\end{tabular}
	\caption{Bezeichnung der Tabelle}
	\label{t1}
\end{table}


\section{Mathematische Formel}\index{Formel}
Mathematische Formeln bzw. Formulierungen k�nnen sowohl im
laufenden Text (z.B. $y=x^2$) oder abgesetzt und zentriert im Text
erscheinen. Gleichungen sollten f�r Referenzierungen nummeriert
werden (siehe Formel \ref{gl-1}).
\begin{equation}
\label{gl-1}
e_{i}=\sum _{i=1}^{n}w_{i}x_{i}
\end{equation}

Entscheidungsformel:

\begin{equation}
\psi(t)=\left\{\begin{array}{ccc}
1 &  \qquad 0 <= t < \frac{1}{2} \\
-1 &  \qquad \frac{1}{2} <= t <1 \\
0 & \qquad sonst
\end{array} \right.
\end{equation}


Matrix:\index{Matrix}
\begin{equation}
A = \left(
\begin{array}{llll}
a_{11} & a_{12} & \ldots & a_{1n} \\
a_{21} & a_{22} & \ldots & a_{2n} \\
\vdots & \vdots & \ddots & \vdots \\
a_{n1} & a_{n2} & \ldots & a_{nn} \\
\end{array}
\right)
\end{equation}

Vektor:\index{Vektor} 

\begin{equation}
\overline{a} = \left(
\begin{array}{c}
a_{1}\\
a_{2}\\
\vdots\\
a_{n}\\
\end{array}
\right)
\end{equation}

\section{S�tze, Lemmas und Definitionen}\index{Satz}\index{Lemma}\index{Definition}

S�tze, Lemmas, Definitionen, Beweise,\index{Beweis} Beispiele\index{Beispiel} k�nnen in speziell daf�r vorgesehenen Umgebungen erstellt werden.

\begin{definition}(Optimierungsproblem)

Ein \emph{Optimierungsproblem} $\mathcal{P}$ ist festgelegt durch ein Tupel
$(I_\mathcal{P}, sol_\mathcal{P}, m_\mathcal{P}, goal)$ wobei gilt

\begin{enumerate}
\item $I_\mathcal{P}$ ist die Menge der Instanzen,
\item $sol_\mathcal{P} : I_\mathcal{P} \longmapsto \mathbb{P}(S_\mathcal{P})$ ist eine Funktion, die jeder Instanz $x \in I_\mathcal{P}$ eine Menge zul�ssiger L�sungen zuweist,
\item $m_\mathcal{P} : I_\mathcal{P} \times S_\mathcal{P} \longmapsto \mathbb{N}$ ist eine Funktion, die jedem Paar $(x,y(x))$ mit $x \in I_\mathcal{P}$ und $y(x) \in sol_\mathcal{P}(x)$ eine
Zahl $m_\mathcal{P}(x,y(x)) \in \mathbb{N}$ zuordnet (= Ma� f�r die L�sung $y(x)$ der Instanz $x$), und
\item $goal \in \{min,max\}$.
\end{enumerate}

\end{definition}

\begin{example} MINIMUM TRAVELING SALESMAN (MIN-TSP)
\begin{itemize}
\item $I_{MIN-TSP} =_{def}$ s.o., ebenso $S_{MIN-TSP}$
\item $sol_{MIN-TSP}(m,D) =_{def} S_{MIN-TSP} \cap \mathbb{N}^m$ 
\item $m_{MIN-TSP}((m,D),(c_1, \ldots , c_m)) =_{def} \sum_{i=1}^{m-1} D(c_i, c_{i+1}) + D(c_m,c_1)$ 
\item $goal_{MIN-TSP} =_{def} min$
\end{itemize}
\begin{flushright}
$\qed$
\end{flushright}
\end{example}

\begin{theorem} Sei $\mathcal{P}$ ein \textbf{NP}-hartes Optimierungsproblem.
Wenn $\mathcal{P} \in$ \textbf{PO}, dann ist \textbf{P} = \textbf{NP}.
\end{theorem}

\begin{proof} Um zu zeigen, dass \textbf{P} = \textbf{NP} gilt, gen�gt es
wegen Satz A.30 zu zeigen, dass ein einziges \textbf{NP}-vollst�ndiges
Problem in \textbf{P} liegt. Sei also $\mathcal{P}'$ ein beliebiges \textbf{NP}-vollst�ndiges Problem.

Weil $\mathcal{P}$ nach Voraussetzung \textbf{NP}-hart ist, gilt insbesondere
$\mathcal{P}' \leq_T \mathcal{P}_C$. Sei $R$ der zugeh�rige
Polynomialzeit-Algorithmus dieser Turing-Reduktion.
Weiter ist $\mathcal{P} \in$ \textbf{PO} vorausgesetzt, etwa verm�ge eines
Polynomialzeit-Algorithmus $A$. Aus den beiden
Polynomialzeit-Algorithmen $R$ und $A$ erh�lt man nun
leicht einen effizienten Algorithmus f�r $\mathcal{P}'$: Ersetzt man
in $R$ das Orakel durch $A$, ergibt dies insgesamt eine polynomielle
Laufzeit. 
%\begin{flushright}
$\qed$
% \end{flushright}
\end{proof}

\begin{lemma} Aus \textbf{PO} $=$ \textbf{NPO} folgt \textbf{P} $=$ \textbf{NP}.
\end{lemma}

\begin{proof} Es gen�gt zu zeigen, dass unter der angegeben
Voraussetzung KNAPSACK $\in$ \textbf{P} ist.

Nach Voraussetung ist MAXIMUM KNAPSACK $\in$ \textbf{PO},
d.h. die Berechnung von $m^*(x)$ f�r jede Instanz $x$ ist
in Polynomialzeit m�glich. Um KNAPSACK bei Eingabe
$(x,k)$ zu entscheiden, m�ssen wir nur noch $m^*(x) \geq k$
pr�fen. Ist das der Fall, geben wir $1$, sonst $0$ aus. Dies
bleibt insgesamt ein Polynomialzeit-Algorithmus. 
\begin{flushright}
$\qed$
\end{flushright}
\end{proof}

\section{Fu�noten}

In einer Fu�note k�nnen erg�nzende Informationen\footnote{Informationen die f�r die Arbeit zweitrangig sind, jedoch f�r den Leser interessant sein k�nnten.} angegeben werden. Au�erdem kann eine Fu�note auch Links enthalten. Wird in der Arbeit eine Software (zum Beispiel Java-API\footnote{\url{http://java.sun.com/}}) eingesetzt, so kann die Quelle, die diese Software zur Verf�gung stellt in der Fu�note angegeben werden.

\section{Literaturverweise}\index{Literatur}
Alle benutzte Literatur wird im Literaturverzeichnis angegeben\footnote{Dazu wird ein sogennanter bib-File, literatur.bib verwendet.}. Alle angegebene Literatur sollte mindestens einmal im Text referenziert werden\cite{Coulouris:02}.
\chapter{Dietrich-Online Projekt}
\label{dietrichonline}

Dietrich online ist eine Datenbank, welche die urheberrechtlich frei gewordenen Dietrich Bände \footnote{Zeitschriften Index von Felix Friedrich Dietrich} durchsuchbar machen soll. Dabei wurden alle Zeitschriften mit einer OCR eingelesen. Zusätzlich werden die Titel der Zeitschriften der Zeitschriften-Datenbank (ZDB) ergänzt und normiert. Auch werden Schlagwörter nach Dewey Decimal Classification (DDC)  \footnote{Klassifikation zur Ordnung von Wissen \cite{DeutscheNationalBibliothek.ddc}} und Gemeinsamen Normdatei (GND)  \footnote{Normdatei für Personen \cite{DeutscheNationalBibliothek.2019b}} an die Zeitschriften angefügt. \cite{UniversityofTrier.2016}
\chapter{Vergleich der Enterprise Search Engines}

In diesem ersten Schritt werden sich diverse Enterprise Search Engine Systeme angeschaut und evaluiert. Dafür wurde eine Liste mit Anforderungen an das System erstellt. Die Anforderungen sehen wie folgt aus:

\begin{itemize}
    \item Open Source oder Kostenlos
    \item Unterstützung von Facetten
    \item Ranking der Suchergebnisse
    \item Volltextsuche
    \item Support für PDF, SQL, XML
    \item Logging-Möglichkeit
\end{itemize}

Des Weiteren sind folgende Features ein Bonus:

\begin{itemize}
    \item Support für PostgreSQL
    \item Backup
    \item statistische Auswertung
    \item Auto-Korrektur und Auto-Vervollständigung
    \item Login System mit Security
    \item bezahlter Support
\end{itemize}

\section{Apache Lucene Core}

Lucene Core ist eine Open Source Enterprise Search Engine von der Apache Foundation geschrieben in Java. Sie bildet die Basis für Solr, auf welches später noch eingegangen wird. Dabei bietet Lucene Core allerdings auch schon einige interessante Features, unter anderem eine Ranked Suche und Facettierung. \ref{vglTable}

Das Lucene Projekt wurde im Jahre 1999 vom Entwicker Doug Cutting gestartet. Es ist im September 2001 der Apache Foundation beigetreten, als ein der Jakarta Familie. 2005 wurde es dann zu einen eigenen Top-Level Projekt. 

\cite{TheApacheSoftwareFoundation.2019b}

\section{Terrier}


\section{Sphinx}
\label{sphinx}


\section{Apache Solr}

Apache Solr ist eine viel eingesetzte Search Engine von der Apache Foundation. Sie basiert auf Apache Lucene Core und erweitert dieses um ein Interface. Die Entwicklung dafür begann 2004 als ein internes Projekt von CNET. Im Jahre 2006 hat CNET den Source Code an die Apache Foundation weitergegeben und es wurde zu einem großen Part des Lucene Projektes. Es wird unter anderem von DuckDuckGo und Best Buy eingesetzt. Durch die Apache Foundation ist ein stabiler Langzeitpartner gegeben. 
Solr arbeitet auf einer REST-like API, welches es ermöglicht eine einfache Schnittstelle für das Dietrich Projekt bereitzustellen. Darüber hinaus bietet es Support für Facetten.
Als Bonus hat es noch Support für Suchvorschläge, was für das Dietrich Projekt auf von Interesse wäre.

Apache Solr hat keinen bezahlten Support, sondern einen kostenfreien Community Support.

\cite{TheApacheSoftwareFoundation.2019}

\section{ElasticSearch}

Eine weitere großes Enterprise Search Engine ist ElasticSearch. Auch dieses Projekt arbeitet auf der Basis von Lucene. Zwei der bekannten Kunden sind Adobe und Ebay. Auch ElasticSearch stellt eine REST-API bereit, allerdings gibt es auch Klienten für viele Programmiersprachen, darunter PHP. Dabei kann auch eine SQL-like Sprache verwendet werden, insofern auf Open-Source verzichtet wird und der Basic Plan genommen wird. \cite{ElasticSearchSub.2019}

Sie bietet den Vorteil, dass Support dazugebucht werden kann. Für eine grafische Oberfläche gibt es Kibana. Dies bietet nicht nur eine Oberfläche für Elastic Search, sondern lässt sich auch mit den Logstash einer Logging-Engine verbinden. Da aktuell auch noch nach einer Logging Lösung gesucht wird, ist dies ein nicht zu unterschätzender Pluspunkt. 

\cite{Elasticsearch.2019}

\section{Manticore Search}

Manticore Search Engine ist eine Open-Source Solution basierend auf Sphinx \ref{sphinx}. Nachdem Sphinx Closed-Source gegangen ist, wurde auf der letzten offenen Version die erste Version von Manticore Search entwickelt. Zu den großen Kunden zählen unter anderem Craigslist und Boardreader.

Manticore erfüllt fast alle Grund Anforderungen, allerdings ist kein nativer PDF-Support gegeben. Es muss daher auf eine Konvertierung der Daten auf XML gesetzt werden. Es findet sich außerdem eine Unterstützung von PostgreSQL, sowie Auto-Korrektur und Vervollständigung. Es gibt auch einen Query-Log. Zuletzt gibt es noch eine Option auf bezahlten Support. Die Supportkosten sind dabei direkt auf der Webseite angegeben und belaufen sich auf 3000 Dollar im Jahr für den Standard Support. 

\cite{ManticoreSoftwareLtd.2019}


\section{Xapian}

Xapian ist eine Open-Source Enterprise Suchmaschine, welche von Zeit-Online, der Universitätsbibliothek Köln und der Debian Webseite genutzt wird. Die Suchmaschine basiert auf Open Muscat, einer Suchmaschine, welche an der Cambridge Universität in den 1980ern von Dr. Martin Porter entwickelt wurde. In 2001, als Open Muscat Closed-Source ging, haben sich einige Entwickler die letzte offene Version geladen und diese weiterentwickelt.

Sie erfüllt alle der Grundanforderungen, wenn auch Logging nur im Grundsinne erfüllt wird, da nur Errors geschmissen werden. Des Weiteren bietet die Suchmaschine Support für PostgreSQL. Auch eine Replikations-Funktion wird mitgeliefert. Sie bietet auch Auto-Korrektur und Auto-Vervollständigung. Ein Login-System mit Sicherheitsfunktionen gibt es durch das Fehlende Frontend Administration nicht. Es gibt allerdings die Möglichkeit mit Omega eine CGI-Suche zu nutzen. Diese Suche bietet allerdings keine Administration, sondern nur eine grafische Oberfläche für Suchanfragen.

Auch gibt es eine Möglichkeit für bezahlten Support. Auf der Webseite werden zwei Firmen angegeben, welche bezahlten Support bieten. Allerdings funktioniert der Link aktuell nur für eine der beiden Firmen aktuell. Zudem ist ein PHP-Connector für die Suchmaschine vorhanden, was die Einbindung ist das Projekt vereinfacht.

\cite{XAP.2019}

\section {Vorauswahl}

\begin{table} %[hbtp]
	\centering
		\begin{tabular}{l | l | l | l | l | l}
		& \textbf{LC} & \textbf{AS} & \textbf{MC} & \textbf{ES}  & \textbf{XP} \\
        \hline
        Open Source                                 & x & x & x & x* & x \\
        Unterstützung von Facetten                  & x & x & x & x  & x \\
        Ranking der Suchergebnisse                  & x & x & x & x  & x \\
        statistische Auswertung (Metriken)          & - & x & x & x  & - \\
        Volltextsuche                               & x & x & x & x  & x \\
        Support für PDF, SQL, XML                   & x & x & x & x  & x \\
        Monitoring / Logging                        & - & x & x & x* & - \\
        Support für PostgreSQL                      & x & x & x & x  & x \\
        Backup                                      & - & x & - & x* & - \\
        Auto-Korrektur                              & x & x & x & x  & x \\
        Auto-Vervollständigung                      & x & x & x & x  & x \\
        Login System und Security                   & - & x & - & x* & - \\
        bezahlter Support                           & - & - & x & x  & - \\
        Wildcard Suche                              & x & x & x & x  & x \\
        Query Vorschläge                            & x & x & x & x  & x \\
        Synonym Support                             & x & x & x & x  & x \\
        PHP Support                                 & - & x & x & x  & x \\
        (Fast) Echtzeit Indizierung                 & x & x & x & x  & x \\
        Web-Interface                               & - & x & - & x  & - \\
        Plugin Support                              & x & x & x & x  & - \\
        JSON Interface                              & x & x & x & x  & - \\
        SQL-Like Query Support                      & x & - & x & x  & - \\
		\end{tabular}
	\caption{Feature-Vergleich der verschiedenen Enterprise Search Engines}
    \label{vglTable}

    Die Tabelle vergleicht einige Features der ausgewählten Search Engines. Dabei wurden die Namen aus Platzgründen wie folgt abgekürzt:

    \begin{itemize}
        \item LC = Lucene Core
        \item AS = Apache Solr
        \item MC = Manticore Search
        \item ES = Elastic Search
        \item XP = Xapian
    \end{itemize} 

    Der Stern bedeutet, dass ein Feature nur in der Kostenlosen und nicht der Open Source Variante verfügbar ist.

\end{table}


Nach einem ersten Überblick wurden nun Aufgrund der Auswahlkriterien diese Systeme zum genaueren Vergleich ausgewählt: Apache Solr, Manticore Search, ElasticSearch (in der kostenlosen Version) und Xapian. Lucene Core wird nicht genauer untersucht, da Solr ein umfassenderes Paket bietet, welches den gestellten Anforderungen mehr entspricht.
\chapter{Genauer Vergleich}

In diesem Kapitel werden die vorher ausgewählten Suchmaschinen genauer verglichen. Dafür werden alle vier Suchmaschinen aufgesetzt und getestet. Hier wird Wert auf alle Aspekte des Prozesses gesetzt. Da ich dieses Projekt nicht nach meiner Bachelor-Arbeit wohl nicht weiter verfolgen kann, ist es auch wichtig zu schauen, wie leicht ein neuer Administrator sich in das System einlernen kann, beziehungsweise wie leicht das System zu verstehen und administrieren ist. Deshalb wird auch die Dokumentation verglichen und geschaut, wie groß die Community der einzelnen Suchmaschinen ist. Die genaueren Kriterien werden nun im Folgenden mit Erklärungen aufgeführt. Da es nicht für jede der Suchmaschinen ein offizielles Docker-Image gibt, werden der Fairness halber alle Systeme manuell aufgesetzt.

Das Test-System hat folgende Spezifikationen:

\begin{itemize}
    \item CPU: 4 Kerne
    \item RAM: 16 Gigabyte
    \item Festplattenspeicher: 20 GB
    \item Betriebssystem: Ubuntu 18.04.03 LTS
\end{itemize} 

Auf das System wird daraufhin die MySQL Datenbank von Dietrich-Online Projekt aufgespielt. Dies für diesen Vergleich die einzige Datenquelle sein.

\section{Aufbau der Tests}

\subsection{Installation}

Im ersten Schritt wird die Installation bewertet, dabei wird geschaut, wie einfach es ist die Software zu installieren. Hierbei ist es wichtig zu schauen, wie simpel die Installation ist. Existiert zum Beispiel ein Installations-Wizard? Wie viel muss manuell in den Dateien geändert werden? Müssen viel externe Programme nachinstalliert werden?

\subsection{Oberfläche}

Als Nächstes folgt der Ersteindruck der Software und Oberfläche. Dabei wird geschaut, wie übersichtlich die Oberfläche ist, falls eine gegeben ist, und wie verständlich das System für Einsteiger ist. Dafür wird im ersten Schritt möglichst auf die Dokumentation verzichtet, um einen Ersteindruck zu liefern, wie gut die Oberfläche für sich selbst spricht. Dies dient dazu um, zu schauen wie der neue Administrator bestimmte Aufgaben ohne Vorkenntnisse erfüllen kann. Besondere Punkte dabei sind zum Beispiel: Wie viel kann man über die Oberfläche konfigurieren? Lassen sich Updates direkt über die Oberfläche einspielen? Ist die Seite responsive? Wie funktioniert die Nutzerverwaltung?

\subsection{Indexierung}

Hier geht es darum festzustellen, wie einfach eine Indexierung der einzelnen Dateien möglich ist. Darunter fällt zum Beispiel: Kann man die Daten über die grafische Benutzeroberfläche indexieren lassen? Kann man das System darauf anweisen Änderungen direkt neu zu indexieren?


\subsection{Dokumentation}

Im dritten Schritt wird die Dokumentation analysiert. Hierbei wird das Augenmerk auf die Übersichtlichkeit und Verständlichkeit gelegt. Auch hier ist es wieder wichtig zu schauen, ob die Dokumentation auch ohne Vorkenntnisse gut zu verstehen ist. Da in diesen Kurztest nicht alle Funktionen durchgetestet werden können, ist es leider auch nicht möglich zu schauen, ob alle Funktionen korrekt und ausführlich dokumentiert sind. Sollte allerdings schon von den Grundfunktionen eine schlechte oder fehlende Dokumentation auffallen wird dies natürlich erwähnt. 

\subsection{Absetzen einer Anfrage und Integration in PHP}

Im letzten Schritt werden einige Query’s abgesetzt. Dies wird zum einen über die Oberfläche geschehen, falls vorhanden, und was wichtiger ist über die Schnittstelle für PHP. Dafür wird ein PHP-Script geschrieben und die Laufzeit des Scripts gemessen.

Der erste Query ist einer der bisher am langsamsten laufenden Query’s. Er ermittelt alle Lemmata vom Buchstaben S und baut alle Daten, die zur Anzeige benötigt werden zusammen \ref{img:errorListSample}. Die Tabellen die für diese Ansicht gebraucht werden, sind in diesen Diagramm \ref{img:lAdminStructure} zu finden. Eigentlich muss man sagen, dass es sich hierbei nicht um einen Query handelt, sondern um zwei. Der erste Sammelt alle ID’s aus der Datenbank, welche unter dem Buchstaben zu finden sind:

\begin{figure}
	\centering
	\includegraphics[width=0.8\linewidth]{images/structure_lemmaadministration.png}
	\caption{Tabellenaufbau der Lemma-Administration Übersicht.}
	\label{img:lAdminStructure}
\end{figure}

\begin{figure}
	\centering
	\includegraphics[width=1\linewidth]{images/lemmaadministration_sample.PNG}
	\caption{Frontend Ansicht der Lemma-Administration mit geladenen Buchstaben S (Ausschnitt).}
	\label{img:lAdminSample}
\end{figure}

\lstset{language=SQL}
\begin{lstlisting}[frame=single] 
    SELECT
    lemma.id
FROM lemma
WHERE
    lemma.bezeichnung LIKE 'S%'
    AND lemma.ist_geloescht = 0
ORDER BY
    lemma.bezeichnung ASC,
    lemma.id ASC;
\end{lstlisting}

Im zweiten Schritt werden dann die gerade geholten ID’s mit vielen JOIN’s für die Darstellung vorbereitet.

\lstset{language=SQL}
\begin{lstlisting}[frame=single] 
SELECT  lemma.id,                        
        [...] #Lemma, GND und DCC-Spalten        
FROM lemma lemma
      
INNER JOIN lemmabearbeitungsstatus lemmaBStatus
ON lemma.fk_lemmabearbeitungsstatus = lemmaBStatus.id

LEFT JOIN lemma_gnd lemma_gnd_map ON lemma.id = lemma_gnd_map.fk_lemma
LEFT JOIN gnd gnd ON lemma_gnd_map.fk_gnd = gnd.id
LEFT JOIN gnd_ddc gnd_ddc_map ON gnd.id = gnd_ddc_map.fk_gnd
LEFT JOIN ddc ddc ON gnd_ddc_map.fk_ddc = ddc.id
WHERE lemma.id IN ([Array of Lemma IDs])
ORDER BY lemma.bezeichnung ASC, lemma.id ASC;

\end{lstlisting}

Der zweite Query ist von dem Fehlermodul \ref{img:errorListSample}. Dieser Query verbindet die drei Tabellen \ref{img:errorListStructure} zu einer Darstellung. Es gibt für jeden Fehler einen Eintrag, dieser wird in tf\_fehler gespeichert. Da es für jeden Eintrag mehrere Fehler geben kann und diese Fehler für jeden Eintrag verfügbar sind, gibt es hier eine n zu m Beziehung. Um diese Abzubilden wurde die Tabelle tfl\_fehler\_rptmap eingefügt. Diese verbindet jeweils die ID’s mit den dazugehörigen Fehlercodes. 

\begin{figure}
	\centering
	\includegraphics[width=0.5\linewidth]{images/structure_errormodule.png}
	\caption{Tabellenaufbau der Fehlerliste.}
	\label{img:errorListStructure}
\end{figure}

\begin{figure}
	\centering
	\includegraphics[width=1\linewidth]{images/errormodule_sample.png}
	\caption{Ansicht der Frontend Tabelle.}
	\label{img:errorListSample}
\end{figure}

\lstset{language=SQL}
\begin{lstlisting}[frame=single]  % Start your code-block

SELECT  fehler.eid,
        [...] #Weitere Spalten von tf_fehler
        fehlercodes.description
FROM tfl_fehler fehler

LEFT JOIN tfl_fehler_rptmap fehler_fehlercode_map 
ON fehler.eid = fehler_fehlercode_map.fk_fehler

LEFT JOIN tfl_rptmap fehlercodes 
ON fehlercodes.id = fehler_fehlercode_map.fk_rptmap

WHERE fehler.band = ('1')
AND fehler.severity > 3
ORDER BY fehler.eid;
\end{lstlisting}

\subsection{Vorbereitung}

Vor der Serverübergabe wurden nur ein paar Hilfsprogramme, wie VIM und Curl installiert. Als Datenbank habe ich dann noch eine MariaDB Instanz aufgesetzt, welche ein aktuelles Datenbank-Abbild von Dietrich-Online erhalten hat.

\section{Solr}

Der erste Kandidat ist Solr. Der Download ist direkt auf der Website ohne Registrierung zu finden. \cite{TheApacheSoftwareFoundation.2019}. Da Solr komplett Open Source ist, kann sich neben den Binary’s auch der Source-Code heruntergeladen werden. 

\subsection{Installation}

Als Systemvoraussetzungen ist eine Java Version $> 8$ gegeben. Ich habe mich hierbei für OpenJDK 11 entschieden. 
Nach dem ersten Starten wurden 2 Warnungen gemeldet, dass gewisse Limits zu gering für Solr sind \ref{lst:warningSolr}. Nachdem beide entsprechend erhöht wurden, verschwanden die Warnungen.

Die Development-Installation ist denkbar einfach. Zuerst wird Solr aus dem Archiv entpackt und dann mit \path{bin/solr start} gestartet. Hierbei wurde ich allerdings direkt von 2 Warnungen begrüßt. 

\begin{lstlisting}[language=bash, frame=single, label={lst:warningSolr}] 
    *** [WARN] *** Your open file limit is currently 1024.
    It should be set to 65000 to avoid operational disruption.

    *** [WARN] ***  Your Max Processes Limit is currently 63918.
    It should be set to 65000 to avoid operational disruption.
\end{lstlisting}

Die richtige Installation installiert Solr als Service und legt sich einen Nutzer an. Ein entsprechendes Installations-Skript findet sich dafür im entpackten Solr-Ordner.

\subsection{Indexierung}

Um mit der Indexierung starten zu können, muss zuerst in sogenannter „Core“ erstellt werden. Dieser ist ein Index mit dazugehörigen Transaktionslog und Konfigurationsdateien. Nur mit diesen ist es möglich Dateien zu indexieren und auf ihnen zu suchen. Nach der Erstellung lässt sich der Core nun auch über die Oberfläche einsehen und zum Teil konfigurieren.

Damit Solr nun die Daten von der Datenbank liest, muss ein DataImportHandler (DIH) \ref{lst:dih} geschrieben werden. In diesen werden die Daten, welche Indexiert werden sollen mit MySQL-Queries eingelesen. Das System basiert dabei auf Entitys. Diese Entitys besizten jedeweils mehre Attribute, wie den Name, welcher auf der Oberfläche zur Indexierung angezeigt wird, den MySQL von dem die Daten geladen werden und einen Delta-Query, welcher nur die neuen Einträge lädt. Der Delta-Query benötigt hierbei eine eigene Spalte in der Datenbank, die einen Timestamp besitzt, der angezeigt, wann die Spalte das letzte mal editiert wurde. Da die Tabellen, diese Spalte aktuell nicht besitzen, wird der Delta-Query nicht getestet werden können. Innerhalb des Entity Elements gibt es entweder weitere Entitys, dazu gleich mehr, oder Field-Elemente. Diese besitzen ein Attribut, welches die Spalte der Tabelle ausweist und einen Namen, der das zugehörige Solr-Schema-Element ausweist. Entitys können unbegrenzt ineinander verschachtelt werden. Damit Änderungen an einer verschachtelten Entity nach oben richtig weitergegeben werden, gibt es ParentDeltaQuerys. Diese geben die betroffenen Werte an die übergeordnete Entity weiter. Dafür führt der ParentDeltaQuery einen Aufruf an die überliegende Entity-Tabelle aus in der er mithilfe der Femdschlüssel-Ids in den betroffen Zeilen herausfindet.

\begin{lstlisting}[language=xml, frame=single, label={lst:dih}, 
    morekeywords={entity,query,deltaQuery,parentDeltaQuery,field,column, name}] 
    <entity name="tfl_fehler" 
            query="select * from tfl_fehler" 
            deltaQuery="select eid from tfl_fehler 
                where last_modified > '${dataimporter.last_index_time}'"> 

		<field column="originalText" name="originalText" />
		<field column="band" name="band" />
        [...] <!-- more Columns -->

      <entity name="tfl_fehler_rptmap"
              query="select fk_rptmap from tfl_fehler_rptmap 
                where fk_fehler='${tfl_fehler.eid}'"
              deltaQuery="select fk_rptmap, fk_fehler 
                from tfl_fehler_rptmap 
                where last_modified > '${dataimporter.last_index_time}'"
              parentDeltaQuery="select eid from tfl_fehler
                where eid=${tfl_fehler_rptmap.fk_fehler}">

        <entity name="tfl_rptmap" [...]> <!-- Queries for rpt_map -->
          <field column="description" name="description" />
        </entity>
      </entity>
    </entity>
\end{lstlisting}

Wie schon eben angesprochen, muss das Solr-Schema für die entsprechende Elemente auch angepasst werden. Dafür gibt es diverse Möglichkeiten, zum einen kann eine XML-Datei angelegt werden, in welcher genau beschrieben wird, wie die einzelnen Einträge indexiert werden sollen. Diese Methode soll allerdings nicht mehr verwendet werden, da es die Möglichkeit gibt, diese Einträge per API einlesen zu lassen. Dabei wird direkt überprüft, ob die Einträge formal stimmen, so können keine fehlerhaften Schema gebaut werden. Dabei werden die Einträge dann in einen Managed Schema gespeichert. 

\begin{lstlisting}[language=xml, frame=single, label={lst:dih}, 
    morekeywords={entity,query,deltaQuery,parentDeltaQuery,field,column, name}] 
    <entity name="tfl_fehler" 
            query="select * from tfl_fehler" 
            deltaQuery="select eid from tfl_fehler 
                where last_modified > '${dataimporter.last_index_time}'"> 

		<field column="originalText" name="originalText" />
		<field column="band" name="band" />
        [...] <!-- more Columns -->

      <entity name="tfl_fehler_rptmap"
              query="select fk_rptmap from tfl_fehler_rptmap 
                where fk_fehler='${tfl_fehler.eid}'"
              deltaQuery="select fk_rptmap, fk_fehler 
                from tfl_fehler_rptmap 
                where last_modified > '${dataimporter.last_index_time}'"
              parentDeltaQuery="select eid from tfl_fehler
                where eid=${tfl_fehler_rptmap.fk_fehler}">

        <entity name="tfl_rptmap" [...]> <!-- Queries for rpt_map -->
          <field column="description" name="description" />
        </entity>
      </entity>
    </entity>
\end{lstlisting}

Dieser muss dann jedoch noch mit dem Core verbunden werden, dafür wird er, zusammen mit einem JDBC-Treiber in die solrconfig.xml eingetragen. Bei den JDBC-Treiber habe ich mich für dieses Beispiel für den Treiber von MariaDB entschieden. Damit es nicht deswegen zu Laufzeit-Unterschieden bei der Indexierung kommen kann, werde ich diesen Treiber bei allen Systeme mit JDBC-DataImportHandlern verwenden.

Als ich den Crawler allerdings startete, meldet mir Solr, das wenige Einträge indexiert wurden. Bei genauerer Betrachtung fand ich heraus, dass nur eine Tabelle von Solr betrachtet wurde. Als ich mir nun die Daten dieser Tabelle herausgegeben habe, bemerkte ich, dass nur das Feld „id“ indexiert wurde. Dies liegt daran, dass nur Felder in den Solr-Index geschrieben werden, die vorher im Schema festgelegt wurden. Dieses Schema kann man entweder über die Weboberfläche oder die API geändert werden kann. Es gibt auch eine Möglichkeit, das Schema direkt zu ändern, allerdings ist diese Methode nicht mehr erwünscht, da die API Fehler erkennt und Einträge so direkt ablehnt. Die so eingelesen Einträge werden in einem sogenannten Managed-Schema \ref{lst:managedSchema} gespeichert. Dort kann man einstellen, wie genau der Eintrag indexiert werden soll. Dabei können viele Einstellungen getroffen werden, um die beste Geschwindigkeit zu garantieren. Für diesen Test wurden die Grundeinstellungen verwendet. Bei der späteren richtigen Implementierung wird auf die einzelnen Felder nochmals genauer eingegangen. {LINK EINFÜGEN GENAUERE DATEN!!!} Aktuell möchte ich nur darauf eingehen, dass man mit "multiValued" eine Einstellung treffen kann, die es erlaubt 1 zu N Beziehungen abzubilden. Zusammen mit der Verschachtelung können so auch M zu N Verbindungen abgebildet werden. 

\begin{lstlisting}[language=xml, frame=single, label={lst:managedSchema}, 
    morekeywords={type,uninvertible,indexed,stored,field,multiValued, name}] 

    [...]
    <field name="ddc_webdewey_is_checked" type="boolean" 
        uninvertible="false" indexed="true" stored="true"/>
    <field name="description" type="text_de" uninvertible="false" 
        multiValued="true" indexed="true" stored="true"/>
    <field name="erweiterung" type="text_de" 
        uninvertible="false" indexed="true" stored="true"/>
    [...]

\end{lstlisting}

Die Indexierung lief eine Minute und 34 Sekunden für 14 Tausend Einträge \ref{img:solrIndexTime}. Dabei wurde der gegebene Arbeitsspeicher nicht komplett ausgenutzt, was schließen lässt, dass die Datenbank der limitierende Faktor war. Es wurden über 435 Tausend Aufrufe gegen die Datenbank gefahren, was darauf zurück zu führen ist, dass Solr keine Joins verwendet, sondern die Datenbank einfach für jede verschachtelte Entity nochmals durchläuft.

\begin{figure}
	\centering
	\includegraphics[width=1\linewidth]{images/solr_indexing_time.png}
	\caption{Startseite der Weboberfläche von Solr.}
	\label{img:solrIndexTime}
\end{figure}

\subsection{Oberfläche}

Der Startseite des Solr-Systems bietet direkten Einblick in auf die Auslastung des Systems. Der Fehler-Log ist auch sehr einfach mit einem Klick zu erreichen. Um Statistiken zu dem aktuellen Core zu bekommen, kann dieser aus einen Drop-Down-Menu einfach ausgewählt werden. Positiv anzumerken ist, dass es möglich ist das Schema direkt in der Weboberfläche zu modifizieren. Leider ist es nicht möglich, den DataImportHandler direkt zu verändern, ohne weitere Einstellungen im System vorzunehmen. Es gibt eine Möglichkeit Querys direkt über den Web-Client zu senden. Auch ist es möglich einen Debug-Modus bei den Query und den DataImportHandler einzuschalten. \ref{img:solrIndexTime} Auch können die Config-Dateien direkt im Browser angeschaut werden. Eine Editierung ist allerdings nicht möglich.
Es gibt keine Möglichkeit Updates direkt über die Weboberfläche einzuspielen, zudem ist die Seite auch nicht responsive geschrieben. 

\begin{figure}
	\centering
	\includegraphics[width=1\linewidth]{images/solr_interface.png}
	\caption{Startseite der Weboberfläche von Solr.}
	\label{img:solrInterface}
\end{figure}


\subsection{Dokumentation}

Die Dokumentation war beim Setup meine Hauptquelle, die Installtion ist sehr genau beschrieben und auch die Systemanforderungen wurden genau beschrieben. Es gibt zu allen Attributen Erklärungen. Allerdings wäre ein Verweis, dass für die DataImportHandler-Attribute noch extra ein Solr-Schema-Attribut benötigt wird, schön gewesen. Dies habe ich erst durch einen Blog \cite{IqubalMustafaKaki.2016} herausgefunden und verstanden. Generell sollte die Dokumentation allerdings zu Administration ausreichen. 

\subsection{Absetzen einer Anfrage und Integration in PHP}

Um den Solr-Client in PHP nutzen zu können muss eine Erweiterung installiert werden. \cite{ThePHPGroup.2019}

\section{Datafari}

\subsection{Installation}

Für Datafari musste folgende Software nachinstalliert werden: Java 8 und JQ, ein JSON-Prozessor. Damit die Installation richtig durchläuft, muss die JAVA\_HOME-Variable erstellt werden. Insofern Datafari nicht unter Root laufen soll, muss noch ein besonderer Nutzer mit Root Rechten angelegt werden. Dieser muss wie schon bei Solr höhere User-Limits erhalten. Datafari installiert sich selbst durch eine DEB-Datei. Während der Installation erscheint ein kurzer Setup-Dialog, welcher einen durch die Konfiguration führt. Das Starten des Server geschieht daraufhin durch ein Script im Installationsordner.

\subsection{Indexierung}

Damit eine Indexierung durchgeführt werden kann, muss bei Datafari ein sogenanntes Repository angelegt werden. In diesem wird die Datenbank-Verbindung eingetragen. Dabei ist es wichtig, dass vorher der Treiber korrekt installiert wird. Es kam bei mir dabei zu Problemen. 
Das auf Apache ManifoldCF basierende System akzeptiert nur MySQL-JDBC Treiber. Da der MariaDB-Treiber einen anderen Klassennamen in Java verwendet, funktioniert dieser nicht. \begin{quote} This connection type cannot be configured to work with other databases than the ones listed above without software changes.~\cite[S.~61]{ApacheSoftwareFoundation.}\end{quote} Deswegen musste ich für diesen Test den MySQL-Treiber von Oracle verwendet.
Nachdem der Treiber korrekt installiert wurde und das Repository erstellt war, kann nun einen Job zu Indexierung der Einträge gestartet werden. In diesem werden die Queries und der Zeitplan konfiguriert.
Im ersten Schritt wird das Repository ausgewählt und das Ziel, in diesem Fall also Solr. In dem Tab Queries lassen sich dann diverse Querys bauen. Der erste ist der Seeding Query, welche eine Art Delta-Query für dieses System ist und natürlich der Data-Query, welcher die Daten aus der Datenbank lädt. Dabei werden mehrere Variablen definiert, damit der Query korrekt von ManifoldCF erkannt wird. Zuerst einmal das Feld: IDCOLUMN, welches die ID enthält, dann URLCOLUMN, welches einen Hyperlink für diesen Eintrag enthält. Da hier keine solche Spalte gegeben ist, wird einfach nochmal die ID mitgegeben, was so in einen Screenshot in der Dokumentation zu sehen ist. Zuletzt noch die DATACOLUMN, welche alle Daten konkateniert enthält. Um das System zu testen habe ich allerdings erstmal nur eine Zeile in die DATACOLUMN geschrieben \ref{img:datafariQuery} Die Konkatenation ist vorgegebene die Methode aus der ManifoldCF-Dokumentation. \cite[S.~97]{ApacheSoftwareFoundation.} Dies ist für unseren Zweck leider keine gute Datenstruktur.
Sind alle Querys eingetragen, kann die Indexierung beginnen. Dafür wird der Job in der Oberfläche manuell gestartet, insofern kein Zeitplan konfiguriert ist. In meinen Test kam es dabei allerdings zu Problemen, die Indexierung erfolgte nicht korrekt und blieb immer am Ende hängen. Der Log zeigte ein „Ready for processing“ an, machte dort allerdings nicht weiter. Einen Eintrag in der Dokumentation oder generell im Internet konnte ich zu diesem Problem nichts finden. Auch eine Reduktion der Einträge auf nur 125 hat das Problem leider nicht lösen können. Deswegen breche ich an dieser Stelle den Test ab. 

\begin{figure}
	\centering
	\includegraphics[width=1\linewidth]{images/datafari_query.png}
	\caption{Übersichtsseite des Querys in Datafari.}
	\label{img:datafariQuery}
\end{figure}

\subsection{Oberfläche}

Die Oberfläche von Datafari ist dreigeteilt. Zum einen gibt eine Such-Oberfläche, welche sich ohne Anmeldung erreichen lässt. Als Zweites findet sich eine Administrationsoberfläche, welche erst eingesehen werden kann, sobald man eingeloggt ist. Dort findet man diverse Einstellungen für die Suchmaschinen, wie Synonyme oder die Facetten-Konfiguration. Auch sind dort die Logs einzusehen, welche durch über Kibana \ref{elasticsearch} angezeigt werden. Die dritte Oberfläche ist die Einstellungsseite für die Datacrawler. Dies ist eine modifizierte Oberfläche von Apache ManifoldCF. Generell sind die Menüs sehr übersichtlich, auch wenn die Einbindung von anderen Anwendungen keine Ideale Lösung darstellt. Es lassen sich keine Updates direkt über die Oberfläche einspielen.
Die Such-Seite und die Seite für die Erstellung der Datacrawler sind Responsive, während die Administrationsoberfläche bei kleineren Bildschirmgrößen das Menü versteckt und die Seite somit unnutzbar macht. Update können auch hier nicht über die Oberfläche eingespielt werden.

\begin{figure}
	\centering
	\includegraphics[width=1\linewidth]{images/datafari_kibana.png}
	\caption{Kibana Integration in Datafari.}
	\label{img:datafariKibana}
\end{figure}

\subsection{Dokumentation}

Die Dokumentation geht sehr genau auf die Installation des Systems ein, dabei werden alle Konfigurationsaspekte beleuchtet. Zum Beispiel wird beschrieben, wie die User Limits erhöht werden, oder die JAVA\_HOME-Variable korrekt gesetzt wird. Allerdings merkt man an manchen Stellen, dass die Dokumentation nicht von nativen Englischsprechenden geschrieben wurde, da die Grammatik nicht immer stimmt. Allerdings hat dies nie zu Problemen oder Verwechslungen geführt.

Bei der Dokumentation zum Einrichten des JDBC-Treibers finden sich einige Probleme \ref{img:datafariJDBC}. Zum einen sind beide Pfade, die in dem Text angegeben sind, falsch. Einer davon wird sogar richtig in dem Screenshot direkt darunter angezeigt. Und zum anderen ist der zweite Screenshot so niedrig aufgelöst, dass sich nicht viel erkennen lässt. Dies passiert auch, wenn das Bild in einen neuen Tab geladen wird. Generell ist die Dokumentation für den Umgang mit Datenbanken nicht sehr ausführlich. Die Erklärungen, wofür die Variablen bei der Erstellung eines Jobs stehen, musste ich in der Dokumentation von ManifoldCF nachlesen.

Die Dokumentation ist im aktuellen Stand nicht sauber strukturiert. Sie gibt das Gefühl, dass es sich eher um eine Sammlung verschiedener Artikel, welche Intern genutzt wurden, handelt.

\begin{figure}
	\centering
	\includegraphics[width=1\linewidth]{images/datafari_doku_wrong_path.png}
	\caption{Dokumentationsseite für den JDBC Treibers von Datafari.}
	\label{img:datafariJDBC}
\end{figure}


\subsection{Absetzen einer Anfrage und Integration in PHP}

Durch das fehlgeschlagene Einlesen der Daten konnte dieser Test leider nicht durchgeführt werden.


\section{ElasticSearch}

\subsection{Installation}

Die Installation ist bei ElasticSearch dreigeteilt. Um ElasticSearch in dem Umfang nutzen zu können, wie es hier gewünscht ist, muss zum einen ElasticSearch, Kibana als auch Logstash installiert werden. ElasticSearch ist hierbei das Kernstück und dient als Datenbank. Kibana ist eine grafische Benutzeroberfläche für ElasticSearch und Logstash stellt die Brücken zwischen der MySQL-Datenbank und ElasticSearch dar. Während ElasticSearch Java mitgeliefert hat, muss für Logstash Java Version 8 oder 11 nachinstalliert werden. Um die drei Dienste für den Development Modus zu installieren, mussten nur die Archive entpackt und die entsprechenden Anwendungen gestartet werden. Ohne die Konfigurationsdateien zu ändern, konnten die Anwendungen direkt miteinander kommunizieren. 

!!Richtige Installtion!!

\subsection{Indexierung}

Um nun Daten zu indexieren, muss in einer Conf-Datei in Logstash definiert werden, wie und welche Daten gelesen und weitergegeben werden sollten \ref{lst:lsConf}. Die Datei kann direkt Querys gegen die Datenbank stellen. Dabei trat zu Beginn allerdings ein Fehler auf, welcher nur damit behoben werden konnte, dass der MariaDB-Treiber direkt im Kern von Logstash mitgeladen wird. Deswegen ist der Pfad zur Treiber-Bibliothek in der Datei auch leer. In den Block Output definiert man nun das Ziel. !!TODO FURTHER!!


\begin{lstlisting}[language=json, frame=single, label={lst:lsConf}] 
  input {
    jdbc {
      jdbc_validate_connection => true
      jdbc_driver_library => ""
      jdbc_driver_class => "Java::org.mariadb.jdbc.Driver"
      jdbc_connection_string =>
          "jdbc:mariadb://localhost:3306/dietrichonline"
      jdbc_user => "USER"
      jdbc_password => "PW"
      statement => "MYSQL-Query"
      schedule => "0 */6 * * *"
    }
  }
  
  output {
    stdout { codec => json_lines }
    elasticsearch {
      document_id => "%{id}"
      document_type => "lemma"
      index => "lemma"
      hosts => "localhost:9200"
    }
  }
\end{lstlisting}


\subsection{Oberfläche}

Es gibt einen Update Knopf!


\subsection{Dokumentation}


\subsection{Absetzen einer Anfrage und Integration in PHP}



\section{Solr}

\subsection{Installation}

Der empfohlene Installationsweg für Xapian führt über das Paketquelle (PPA) der Entwickler. Nachdem dieses eingefügt wurde, kann Xapian entweder in einer C++ oder Python Variante installiert werden. Damit Xapian auch mit PHP anzusprechen ist, ist es allerdings notwendig den PHP-Connector aus Lizenzgründen selbst zu bauen. Dabei muss vorher der ein Eintrag, der ausweist, dass aus dieser Quelle auch Source-Code geladen werden kann, in der Paketquellen hinzugefügt werden. Nachdem nun der PHP-Klient gebaut worden ist, kann dieser nun normal installiert werden.
Allerdings ist der Server bisher nur Lokal ansprechbar, um dies zu ändern, muss ein TCP-Server für Xapian gestartet werden. Um diesen zu Nutzen ist es vonnöten ein weiteres Paket aus der PPA zu installieren. Damit nun der Server gestartet werden kann, muss zuerst ein Index, der bei Xapian Database genannt wird, gebaut werden. Dazu mehr in dem Teil \ref{xap:index}. Danach kann der Server auf einen beliebigen Port gestartet werden.

\subsection{Indexierung}
\label{xap:index}

Durch die fehlende Dokumentation zur Indexierung von MySQL-Datenbanken, habe ich zuerst einmal ein gegeben Beispiel zur Indexierung von eine CSV Datei durchgearbeitet. Dabei ist mir aufgefallen, dass zur Indexierung ein Parser für MySQL-Daten zu Xapian komplett selbst geschrieben werden muss. Soll nun also eine Datenbank indexiert werden, muss die Datenbank selbst durchsucht und die einzelnen Felder an Xapian weitergegeben werden.  

Um nun die Daten von der Datenbank in Xapian zu indeixeren, muss nun eine Sprache in der es möglich ist eine Datenbankverbindung aufzubauen benutzt werden. Ich hab mich dabei für PHP entschieden. Die Indexierung läuft dabei folgendermaßen ab. !!!!

Als nun das PHP-Script auf den Server gestartet wurde, mussten noch einen Warnungen und Fehler behoben werden. Dazu wurde die php.ini angepasst. 

$PHP WARNING -> PHP Warning:  dl(): Dynamically loaded extensions aren't enabled in /usr/share/php/xapian.php on line 22$


\subsection{Oberfläche}

Xapian besitzt keine Oberfläche zur Verwaltung. Allerdings kann sich ein Such-Frontend installiert werden, welches allerdings hier nicht geprüft wurde, da beim Dietrich sowieso ein eigenes Frontend gebaut werden musste.

\subsection{Dokumentation}

Bei dem letzten Versionsupgrade wurde die Dokumentation von Xapian komplett umgeschrieben. Diese neue Dokumentation hat bisher noch viele Lücken und besitzt auch Todo-Boxen \ref{img:xapianDoku}.
Zu der Installation von dem TCP-Server war nichts in der neuen Dokumentation zu finden. Durch Googlen, wie der Server extern ansprechbar gemacht werden kann, bin ich zufällig auf eine Seite der Dokumentation gestoßen, welche einen Befehl zum Starten vermerkt hatte. Nachdem dieser Befehl ausgeführt wurde, wurde gemeldet, dass für diesen Befehl ein weiteres Paket installiert werden musste. Dieses Paket wurde in der Dokumentation nicht genannt. 

Generell bietet die Dokumentation bisher nur einen sehr grundlegenden Einblick in das System und beleuchtet keine genauen Themen, wie das Indexieren von MySQL-Datenbanken. Positiv anzumerken ist allerdings, dass Xapian ein Beispiel zu Indexierung von Daten mit Code bereitstellt. Allerdings ist dieses nur in Python verfügbar.

\begin{figure}
	\centering
	\includegraphics[width=1\linewidth]{images/xapian_doku.png}
	\caption{Screenshot von der Xapian-Dokumentation}
	\label{img:xapianDoku}
\end{figure}


\subsection{Absetzen einer Anfrage und Integration in PHP}



\chapter{Fazit des Vergleiches}

Nachdem nun alle Systeme für einen Ersteindruck aufgesetzt worden sind, ist es nun an der Zeit, eine Suchmaschine auszuwählen. Dazu wurde ein Treffen mit einigen Mitarbeitern der Bibliothek einberufen, um die Ergebnisse zu diskutieren. 

Dabei wurde schnell Datafari durch die Probleme, welche beim Test auftraten, ausgeschlossen. Auch Xapian wurde durch die schwache Benutzerfreundlichkeit und der unfertigen Dokumentation abgelehnt. 

Verbleibend waren nun noch Solr und Elasticsearch. Entschieden wurde sich dann letztendlich für Elasticsearch. Die lag daran, dass Elasticsearch viele Sicherheitsfunktionen direkt mitliefert und wohl auch bald in der Universitätsbibliothek Köln verwendet wird. Zudem bietet Elasticsearch die einsteigerfreundliche Erfahrung des Testes. Daher wird nun im nächsten Kapitel mit Elasticsearch weitergearbeitet.
\chapter{Nutzung des Open Archives Initiative Protokolls für Metadaten}

Während des Projektes kam die Frage auf, ob das Open Archive Initiative Protokoll für die Datenerhaltung verwendet werden sollte. Daher wird dies im folgenden kurz geprüft.


\section{Open Archives Initiative Protokolls}

Das Open Archives Initiative Protocol for Metadata Harvesting (OAI-PMH) ist ein Protokoll zum Austausch von Metadaten. Dabei werden Anfragen per GET oder POST-Request angefragt. Als Antwort erhält man im Folgenden ein XML-Dokument. So können die Metadaten mit bestimmten Facetten abgefragt werden (zum Beispiel Autor). Dabei geht es allerdings darum primär darum Änderungen weiterzugeben. So können durch dieses Protokoll neue Einträge oder Änderungen in der Datenbank weitergeben werden.
\cite{DeutscheNationalBibliothek.2019}

\section{OAI Harverester}

Ein OAI Harverester ist ein Programm, welches durchgehend einen Abgleich der Daten vollführt. Dabei lässt es sich die Änderungen mit einem List-Befehl von dem Server geben und gleicht diese danach mit der eigenen Struktur ab. Sollten dabei Unterschiede festgestellt werden, werden daraufhin die Änderungen auch beim Harverester eingefügt. So steht der Harverester immer mit dem Server auf einen Stand.
\cite{DeutscheNationalBibliothek.2019}

\section{Support der Enterprise Search Engines}

Bei den vorhin genannten Enterprise Search Engines gibt es keine mit nativen OAI Harverester Support. Es gibt die Möglichkeit für manche der Suchmaschinen ein solches Verhalten mithilfe von Plugins zu implementieren. Allerdings sind die meisten dieser Add-ons auch schon veraltet.

\section{Auswertung}

Durch eine fehlende Basisimplementierung des Protokolls in den einzelnen Suchmaschinen und der Möglichkeit eines direkten Zugriffs auf die Datenbank, sehe ich keinen Grund dieses Protokoll zu verwenden. Es müsste ein Server vor die Datenbank installiert werden und ein Harverester vor der ESE. Dies ist ein großer Mehraufwand, welcher bei diesem Anwendungsfall nicht notwendig ist. Sollte allerdings diese Suchmaschine ein übergreifendes System werden, kann darüber nachgedacht werden, die anderen Datenbanken per OAI-Harverester anzusprechen.
\chapter{Setup}

Diese Kapitel handelt von der Installation und Ersteinrichtung der Suchmaschine über Docker. Bei einem Meeting wurde beschlossen, dass ElasticSearch als Suchmaschine genutzt wird. Die Installation erfolgt dabei über Docker-Compose, zum Verwalten von mehreren Docker-Maschinen. Dabei werden 2 ElasticSearch Instanzen, eine Kibana Instanz und eine Logstash-Instanz verwendet. Um das System abzusichern wird zudem der Reverse-Proxy Traefik verwendet. Dabei wird Kibana mit SSL-Verschlüsselung und dem Login-System versehen. Auch die Anfragen an ElasticSearch werden mit einer Basic-Authenification gesichert.

\section{Docker}

Die Konfiguration der Instanzen funktioniert größtenteils über YAML-Datei. Dabei wird für jedes System die entsprechende Datei gemountet. 

Zudem möchte ich noch gern auf die Rechte-Verwaltung von Docker eingehen. Jeder Container kann seine eigenen Nutzer anlegen, jedoch können diese nicht unbedingt auf den Filesystem der Hosts schreiben. Um nun einen Docker-Container Nutzer das Recht zu geben auf den Filesystem zu schreiben, muss die User-ID der Datei auf denselben Wert gesetzt werden, wie der Nutzer im Container. ElasticSearch verwendet grundsätzlich einen Nutzer mit der UID 1000 und der GUID 1000. Deswegen müssen die Ordner auf dem Host-System nun auch diese Rechte bekommen. Dazu wurden mit CHOWN die Rechte auf 1000:1000 gesetzt. Allerdings ist der Nutzer 1000 auf dem Host-System ein anderer Nutzer. Deswegen besitzt jetzt ein anderer Nutzer die Dateien in den ElasticSerach Ordner. Um dies zu beheben, wurde der Nutzer auf eine andere UID/GUID gesetzt um Konfusion zu vermeiden. \cite{JarrodWeaver.2014}

\subsection{ElasticSearch}

\begin{lstlisting}[language=XML, frame=single, label={lst:es01}] 
	es01:
	image: docker.elastic.co/elasticsearch/elasticsearch:7.5.1
	container_name: es01
	environment:
		- "ES_JAVA_OPTS=-Xms4g -Xmx4g"
	ulimits:
		memlock: -1
	volumes:
		- /srv/elk/elasticSearch01/:/usr/share/elasticsearch/data
		- /srv/elk/config/elasticsearch.yml:
			/usr/share/elasticsearch/config/elasticsearch.yml
	ports:
		- 9200:9200
	networks:
		- elastic
\end{lstlisting}

Für die beiden ElasticSearch Instanzen wurde zudem der Java-Speicher erhöht. Es werden 4 Gigabyte benutzt, da wir 2 Server haben, welcher allerdings in Kombination niemals mehr Speicher verwenden sollen als 50 \% des gesamten RAMs. \cite{ElasticsearchB.V..12172019}

Der ulimits Befehl hebt die Begrenzung des Memory-Locks auf, damit ElasticSearch korrekt arbeiten kann.

Als Volumes ist zum einen die oben genannte YAML-Datei angegeben und zum anderen wird der Datenordner gemountet. Dies dient dazu, dass, falls der Container zerstört wird, die indexierten Daten trotzdem weiterhin gespeichert werden.

Der Port wird zum Host-System durchgereicht, damit das System auch von außerhalb des Docker-Netwerkes zu erreichen ist. Dabei ist das System troz blockierter UFW zu erreichen. Dies liegt daran, dass Docker in der Standardeinstellung UFW ignoriert.

In der ElasticSearch-Konfigurationsdatei werden nun die Einstellungen, die speziell für das ElasticSearch-System relevant sind verwaltet. 

Darin wird zuerst der Cluster-Name definiert. Dieser dient dazu, dass die Server wissen, dass Sie dieselben Daten betreuen. 
Danach wird der Name des Servers vergeben. Dieser wird für spätere Einstellungen noch wichtig.

Das Memory-Lock Setting dient dazu, dass die Anwendung verhindert, dass sie in den SWAP gelegt wird. Laut ElasticSearch-Website bricht die Suchgeschwindigkeit ein, sollte das Programm in den SWAP gelegt werden.

Der Network Host wird hier auf alle Interfaces der Maschine gesetzt, damit sich alle System innerhalb der Docker Netzwerkes finden können.
Das Seed-Host Setting sagt aus, an welche Nodes die Daten synchronisiert werden sollen.

Der letzte Eintrag dient dazu, dass bei der ersten Synchronisation das System weiß, welche Nodes alle Daten enthalten, also mit welchen Server sich synchronisiert werden soll. Da hier beide Systeme beim ersten Start noch keine Daten besitzen, sind alle Nodes zu beginnt Master. 
\begin{lstlisting}[language=XML, frame=single, label={lst:es01-yml}] 
	cluster.name: dietrich-online-cluster
	node.name: es01
	bootstrap.memory_lock: true
	network.host: 0.0.0.0
	discovery.seed_hosts: ["es02"]
	cluster.initial_master_nodes: ["es01", "es02"]
\end{lstlisting}

\subsection{Kibana}

Die Grundkonfiguration von Kibana ist einfacherer als die Konfiguration von ElasticSearch. Es muss nur die YAML-Datei gemountet werden und der Port 5601 nach außen durchgereicht werden.

In der YAML-Datei werden nun die Einstellungen für Kibana gesetzt. Darunter fällt der oben genannte Port, der Server-Host, in diesem Fall auch 0.0.0.0, damit Kibana auf alle Interfaces innerhalb des Containers hört, und die ElasticSearch-Hosts. Dabei werden alle Server Instanzen mitgegeben, auf denen Kibana arbeiten soll. 

\subsection{Logstash}

Für die Grundkonfiguration von Logstash muss, wie schon im Test vorher, der Treiber in die Core-Bibliothek gemountet werden. Zudem wird ein Ordner gemountet, in dem alle Conf-Dateien für die Pipeline geladen werden. Dieser Ordner wird dann alphabetisch von Logstash abgearbeitet. In der YAML wird dann der Name, die Pipeline.id und die Pipeline-Worker festgelegt. Die Pipeline-Worker sind die Threads in denen eine der eben genannten Conf-Dateien bearbeitet wird. Generell sollte die Anzahl der Cores auch die maximale Anzahl der Worker sein.


\section{X-Security}

X-Security nennt sich das Paket mit den Sicherheitseinstellungen vom ELK-Stack. Dafür habe ich hier den kompletten Traffic zwischen den einzelnen Komponenten, sowie der Endnuzter zum Server mit SSL verschlüsselt. Dazu mussten zuerst einmal die Zertifikate generiert werden. Dafür bietet ElasticSearch ein Tool an, welches eine Autorität und die einzelnen Zertifikate mit Private und Public-Key generiert. Allerdings werden diese standardmäßig im PKCS 12-Format (FOOTNOTE) abgespeichert. Dieses ist ein Container-Format, welches die Schlüssel und die Autorität zusammen verpackt. Jedoch benötigt Kibana zum Beispiel nur die Autorität als einzelnes Zertifikat und nicht in einem Container.

Es gibt eine Möglichkeit diese Dateien aus der PK12 zu entpacken, jedoch gab es auch hierbei Probleme, da OpenSSL, das Tool welches zum Entpacken verwendet wird, die Autorität nicht richtig entpackt. \cite{nerophon.2018}

Es kann allerdings auch bei der Zertifikat-Erstellung eine Option mitgegeben werden, dass die Zertifikate nicht verpackt werden sollen. Um nun alle Zertifikate gleichzeitig zu generieren, kann eine YAML Datei mitgegeben werden, welche alle Informationen zu den einzelnen Systemen enthält. In diesem Fall wurde nur der DNS-Name angegeben \ref{lst:certs-yml}. Mit diesen Informationen wurde nun für alle Systeme Zertifikate generiert. 

\begin{lstlisting}[language=XML, frame=single, label={lst:certs-yml}] 
	instances:
	- name: 'es01'
	  dns: [ 'es01', 'bib55', 'bib55.uni-trier.de' ]
	  [...]
\end{lstlisting}

Um die Zertifikate nun einzutragen, mussten nur in den diversen YAML-Dateien alle Pfade angegeben werden. 

Zusätzlich zu den Zertifikaten muss auch noch eine Password Authentifikation eingebaut werden. Dazu kann auf den ElasticSearch-Containern ein Befehl zur Erstellung der Systempasswörter aufgerufen werden. Diese müssen dann auch in den YAML-Dateien vermerkt werden. Weitere Nutzer können dann per API oder Kibana erstellt werden. Die Verteilung der Rechte passiert hierbei rollen-basiert. Es wird zuerst eine Rolle erstellt, mit den gewünschten Rechten, welche daraufhin an den Nutzer weitergegeben wird.

Um nun einen Query gegen das ElasticSearch System zu stellen, muss zum einen eine BasicAuth sowie das Zertifikat mitgegeben werden \ref{lst:curlQuery}.

\begin{lstlisting}[language=BASH, frame=single, label={lst:curlQuery}] 
	curl https://bib55:9200 --cacert ca.crt -uuser:pass
\end{lstlisting}

Um nun die Daten mit Logstash einzuspielen, wurde ein Nutzer erstellt, welcher nur Zugriff auf Indices mit dem Präfix dietrich\_ und die Logstash-Indices besitzt. Diese Einstellungen konnten alle über die Benuzter-Oberfläche von Kibana gemacht werden \ref{img:kibanaRoles}. 


\begin{figure}
	\centering
	\includegraphics[width=1\linewidth]{images/setup/kibana_roles.png}
	\caption{Seite zu Erstellung von Rechte-Rollen}
	\label{img:kibanaRoles}
\end{figure}

\subsection{Vollständiges Setup}

GRAFIK
\chapter{Implementation in das DietrichOnline-Projekt}

In diesem Kapitel wird die Implementation in das DietrichOnline-Projekt genauer erläutert.

\section{Vorbereitung}

Um den Elasticsearch-Klienten zu nutzen, muss dieser zuerst einmal per Composer installiert werden. Dies geschieht durch einen Eintrag in der composer.json.

Um einen möglichst genauen Vergleich liefern zu können, wurde eine Datenbank, welche auf demselben Server wie Elasticsearch liegt, verwendet. So ist sichergestellt, dass die Verbindung zum Server für beide Systeme gleich ist.

Zum Testen wird nun zuerst die Lemma-Abfrage, in einer modifizierten Fassung \ref{lemmaIndexierungEla}, verwendet.

Durch die aktivierte X-Security von Elasticsearch braucht der Klient einen Zugang per API-Key. Dieser kann mithilfe eines curl-Befehls generiert werden, siehe Listing \ref{lst:elaApi}. Dieser API-Key bekommt hierbei lesende Rechte auf alle Dietrich-Indices. Auch muss die Zertifikats-Autorität bei jeder Anfrage mitgegeben werden:

\begin{lstlisting}[language=JSON, frame=single, label={lst:elaApi}, caption=Code zur Generierung eines API-Keys,captionpos=b] 
POST /_security/api_key
{
  "name": "dietrich-website",
  "role_descriptors": { 
    "role-a": {
      "cluster": ["all"],
      "index": [
        {
          "names": ["dietrich_*"],
          "privileges": ["read"]
        }
      ]
}}}
\end{lstlisting}


\subsubsection{Indexierung}
\label{lemmaIndexierungEla}

Um nun alle Daten in richtiger Form in das Projekt zu laden, muss die Indexierung von dem Ersteindruck \ref{elaVgl:index} umgeschrieben werden. 

Bei den SQL-Joins wurden M:N-Beziehungen auf eine flache Ebene gezogen. Dabei wird der Eintrag so oft abgebildet, wie es Objekte in der M:N-Beziehung gibt. 

Zur Verdeutlichung hier ein Beispiel.
Es gibt eine Tabelle Artikel, welche eine M:N-Beziehung mit der Tabelle DDC bezieht. In der Artikel Tabelle gibt es den Eintrag Trier mit der ID 1, der mit 2 DDC Einträgen, Trier und Rheinland-Pfalz, verbunden wird. Bei einem SQL-Join werden beide referenzierten Beiträge auf eine flache Hierachie gezogen. Deswegen würde die Tabelle der Ergebnisse des SQL-Joins nun die folgenden Einträge enthalten \ref{tbl:join}.

\begin{table} %[hbtp]
	\centering
		\begin{tabular}{l | l | l}
		    \textbf{ID} & \textbf{Artikel} & \textbf{DDC} \\
        \hline
        01 & Trier & Trier \\
        01 & Trier & Rheinland-Pflaz  \\
		\end{tabular}
    \caption{Tabelle für ein Beispiel der SQL-Joins}
    \label{tbl:join}
\end{table}

Um nun einen solchen Eintrag in Elasticsearch abzubilden, wird ein Array benötigt. Für solche Fälle gibt es den Aggregat-Filter in Logstash. Dieser aggregiert auf Basis der ID die Daten. So ist es möglich, Code zu schreiben, der automatisch die Daten in Arrays zusammenfasst:

\begin{lstlisting}[language=Ruby, frame=single, label={lst:fronendConf}, caption=Auschnitt aus dem Aggregat-Filter von Logstash,captionpos=b] 
    [...]
    map['bstatus_beschreibung'] ||= event.get('bstatus_beschreibung')
    
    map['ddc_entries'] ||= []
    if event.get('ddc_notation') != nil
        duplicate = false
        map['ddc_entries'].each { |n|
            if n.value?(event.get('ddc_notation'))
                duplicate = true
                break
            end
        }
        if !duplicate
            map['ddc_entries'] << {
                'ddc_notation' => event.get('ddc_notation'),
                'ddc_schlagwort' => event.get('ddc_schlagwort'),
                'ddc_webdewey_is_checked' => 
                  event.get('ddc_webdewey_is_checked')
            }
        end
    end
    [...]
    \end{lstlisting}

Dabei ist es wichtig, dass dieser Prozess nicht in mehreren Threads ausgeführt wird. Daher erhält jede Pipeline in diesem Projekt auch maximal einen Thread.

Dieser Code wird nun so lange in einer Schleife durchlaufen, wie dieselbe ID aus der Datenbank kommt. Der Wert bstatus\_beschreibung in Zeile 2 wird dabei zum Beispiel bei jedem Durchlauf überschrieben. 

Damit die sich ändernden Wert aggregiert werden, werden diese in ein Array geschrieben. Da bei der Abfrage \ref{lst:sqlQuery} mehrere SQL-Joins aufeinander ausgeführt werden, kann es dazu kommen, dass sich Zeilen wiederholen. Um dieses Problem zu lösen, werden Duplikate und NULL-Werte erkannt und nicht nochmals in das Array einträgt.

\section{Aufbau der Abfrage}

Als SQL-Framework wurde Doctrine verwendet. Dieses bietet eine Abstraktion für SQL in Objekte. Im Hintergrund werden diese Objekte dann in SQL übersetzt. Die hier betrachtete Abfrage wurde schon von Hand optimiert, da zuerst alle Ids der anzuzeigenden Lemmata gesucht werden, und erst im zweiten Schritt alle SQL-Joins auf den verbleibenden Datensätzen ausgeführt werden.

Elasticsearch arbeitet mit nur einer Abfrage, da hier alle Daten schon auf einer flachen Ebene existieren.

Verwendet wird hier ein sogenannter Boolean-Query. Dieser enthält vier verschieden Untergruppierungen.

Zuerst einmal der Must-Teil. Alle hier angegeben Parameter müssen in jedem Ergebnis vorhanden sein. Dies ist gleichzustellen mit einem booleschen AND. 

Als Zweites der Must-Not-Teil. Dieser Teil ist dem Must-Teil sehr ähnlich, allerdings sind die Parameter negiert.

Danach der Should-Teil. In diesem Teil muss nur einer der Parameter vorhanden sein. Dies ist zu vergleichen mit einem booleschen OR. 

Und zuletzt der Filter-Teil. Die gesetzten Filter sind auch Must-Befehle. Allerdings werden diese nicht bei der Gewichtung der Ergebnisse mit eingerechnet. Für diese Arbeit hat dies erstmal keinen Einfluss, da alle Ergebnisse alphabetisch sortiert werden. Daher ist diese Abfrage mit Must gleichzustellen. \cite{ElasticsearchB.V..17.12.2019}


\begin{lstlisting}[language=PHP, frame=single, label={lst:queryEla}, caption=Abfrage an Elasticsearch in PHP,captionpos=b] 
//Create Client with basic Params
$mustNotQueries = [];
$filters        = [];
$mustQueries    = [];

if ($character === LemmaEntity::NOT_A_TO_Z_CHARACTER) {
    $mustQueries[] = ['regexp' => ['bezeichnung.keyword' => 
      ['value' => '@&~(^[a-zA-Z].+)', 'flags' => 'ALL']]];
      
    $filters[]     = ['term' => ['ist_geloescht' => false]];
} elseif ($character === LemmaEntity::DELETED) {
    $filters[] = ['term' => ['ist_geloescht' => true]];
} else {
    $mustQueries = ['prefix' => ['bezeichnung.keyword' => "$character"]];
    $filters[]   = ['term' => ['ist_geloescht' => false]];
}

switch ($filter) {
    case self::STATUS_FILTER_KLAR:
        $filters[] = ['term' => ['bstatusbezeichnung' => 'klar']];
        break;
    //[Other Filters]
}

$params['body']['query']['bool']['must']     = $mustQueries;
$params['body']['query']['bool']['must_not'] = $mustNotQueries;
$params['body']['query']['bool']['filter']   = $filters;

return $client->search($params)['hits']['hits'];
\end{lstlisting}

In Zeile 14 ist hier zu sehen, dass anstelle der Wildcard-Abfrage, welcher in der vorherigen Abfrage im Kapitel \ref{lst:phpElastic} verwendet wurde, ein Prefix-Abfrage verwendet wird. Dieser bietet eine sauberere Lösung zum Suchen von Wortanfängen.

\section{Vergleich von der Geschwindigkeit der Abfragen}

Die oben beschriebenen Abfragen wurden jetzt jeweils 100-Mal mit einem Timer laufen gelassen. 

Bei dem Vergleich kamen die folgenden Durchschnittswerte zustande:
\begin{table} %[hbtp]
	\centering
		\begin{tabular}{l | l }
		    \textbf{System} & \textbf{Zeit} \\
        \hline
        MariaDB + Doctrine & 3.49 \\
        Elasticsearch      & 1.45  \\
		\end{tabular}
    \caption{Vergleich der Laufzeit zur Abfrage aller Daten für den Buchstaben S der Lemma-Administration (15.846 Einträge)}
    \label{vlgTimeDBvsEla}
\end{table}

Dabei ist zu sehen, dass Elasticsearch eine Reduktion der Zeit um 58,45 \% ermöglicht.

Diese Abfrage wurde auch noch in einer nachgebauten Produktionsumgebung ausgeführt. 

\begin{figure}
	\centering
	\includegraphics[width=1\linewidth]{images/setup/query/time_prod_ela.png}
	\caption{Geschwindigkeit: Elasticsearch}
	\label{img:timeProdEla}
\end{figure}

Daran erkennt man, dass die Abfrage durchaus schneller agiert, allerdings im Gesamtkontext kaum einen Unterschied macht. Elasticsearch lädt mit 46,49 Sekunden durchaus schneller als Doctrine mit 50,1 Sekunden, allerdings fällt dies bei der langen Gesamtladezeit kaum ins Gewicht.

\begin{figure}
	\centering
	\includegraphics[width=1\linewidth]{images/setup/query/time_prod_db.png}
	\caption{Geschwindigkeit: Doctrine}
	\label{img:timeProdDb}
\end{figure}


\chapter{Frontend-Suche}

\section{Indexierung}

Um die Daten zu indexieren, wurde eine statische Code-Analyse durchgeführt, welche Daten alles im Frontend angezeigt wurden. Dafür wurde der gesamte Code zur Suche untersucht und alle Werte aufgeschrieben. Aufgrund dieser Basis wurde daraufhin ein Query gebaut, welche alle Daten an den Aggregat-Filter weiterreicht. Dabei wurde dieser angepasst und enthält nun auch eine String Konkatenation, welche zuvor bei der Anzeige ausgeführt wurde.

Der Index ist hierbei der aktuell Größte im Dietrich-Online Projekt mit rund 1.4 Millionen Einträgen. Diese Größe des Index beläuft sich auf 2.4 Gigabyte.

Diese Zahl wird sich allerdings verringern lassen, wenn mehr Regeln zu Indexierung eingebaut werden. Aktuell werden alle Spalten mit Text, die unter 256 Zeichen fallen einmal als Keyword und als Volltext indexiert \ref{elaVgl:index}.

Der Query enthielt nun 13 Joins. Diese Datenmenge konnte Logstash nicht verarbeiten und stürzte ab. Der RAM musste auf 4 Gigabyte erhöht werden, damit der Index ordentlich aufgebaut werden konnte.

Damit nun verschiedene Pipelines gleichzeitig arbeiten können, musste eine Datei zur Verwaltung der Pipelines angelegt werden. In dieser wurde definiert, dass von den vier verfügbaren Workern jeweils einer pro Pipeline zur Verfügung steht. 

\section{Integration}

Die Integration folgte demselben Muster, wie die der Lemma Administration. Ich möchte hier allerdings auf ein paar Unterschiede eingehen. 



\subsection{Paginierung}
Zum einen werden diesmal 2 Querys abgeschickt. Zum einen ein Count-Query und zum anderen ein Query mit den Ergebnissen. Dies liegt daran, dass die Paginierung direkt zu Beginn die Anzahl der Ergebnisse wissen muss. 

Bisher wurde die Paginierung vom Symfony übernommen, es wurden maximal 1001 Ergebnisse aus der Datenbank geholt und diese dann paginiert. Nun wird die Paginierung von ElasticSearch verwendet. Dort kann bei einem Query ein Offset mitgegeben werden, welcher es ermöglicht mehr Ergebnisse zu holen, ohne Einbußen in der Performance zu haben. 

\subsection{Boolesche Logik}

Zum anderen ist es möglich eine boolesche Logik bei der Suche zu verwenden. Um diese Umzusetzen, werden die Query-Teile ineinander verschachtelt \ref{lst:booleanEla}. 

Bei jeder Suche wird ein Array mit allen Suchanfragen weitergegeben. Das erste Item in Array hat dabei niemals einen Junktor. Dafür existiert der erste Fall. Existiert eine weitere Stelle im Array ist auch ein Junktor mit angegeben. Dieser wird dann in dem unten gezeigten Switch-Case ausgelesen. Dann wird ein weiterer Boolean-Query geschrieben, welcher zum einen den zweiten Teil der Suche, sowie die bisherige Suche enthält.

\begin{lstlisting}[language=PHP, frame=single, label={lst:booleanEla}] 
    switch ($userSearchItem->getJunktor()) {
        case UserSearchItem::JUNKTOR_NO: //First Entry
            $this->fullQuery = [
                'bool' => [
                    'must' => [
                        $this->addTypeValue($userSearchItem), //Add Search
                    ],
                ],
            ];
            break;
        case UserSearchItem:: JUNKTOR_AND: //MUST
            $this->fullQuery = [
                'bool' => [
                    'must' => [
                        $this->addTypeValue($userSearchItem), //Add Search
                        $this->fullQuery, //First Part of Query
                    ],
                ],
            ];
            break;
    [...] // More Cases like OR or AND NOT
\end{lstlisting}
\chapter{Zusammenfassung und Ausblick}

Diese Bachelorarbeit hat sich ausführlich mit Enterprise Suchmaschinen auseinandergesetzt, diese Verglichen und letztendlich eine in das Dietrich Online Projekt implementiert. 

Im ersten Schritt wurden diverse Suchmaschinen erstmal nach einer Anforderungsliste verglichen. Dafür wurde eine Tabelle erstellt, welche alle Suchmaschinen anhand der gefundenen Funktionen verglichen. Mithilfe dieser Basis wurden vier Suchmaschinen für den genaueren Vergleich herausgesucht.

Für den genaueren Vergleich wurden diese Suchmaschinen nacheinander aufgesetzt und einige Dokumente indexiert. Dabei musste die Suchmaschine selbständig die Daten aus der Suchmaschine laden und indexieren. Zudem wurde auch die Benutzerfreundlichkeit untersucht. Dafür wurde die Oberfläche, insofern eine vorhanden war, und die Dokumentation bewertet. Zum Schluss wurde daraufhin eine Suchmaschine ausgewählt, welche in das Dietrich Online Projekt implementiert werden sollte. Dabei war es aufgrund der Zeit leider nicht möglich einen korrekten wissenschaftlichen Vergleich zu erstellen. Es wurde lediglich ein Ersteindruck gewonnen.

Als Nächstes wurde über die Möglichkeit nachgedacht einen OAI Harverester vor die Datenbank zu stellen, um eine normierte Schnittstelle zwischen der Datenbank und Suchmaschine herzustellen. Nach einer kurzen Analyse wurde diese Methodik allerdings verworfen, da ein direkter Zugriff auf die Datenbank möglich ist und somit der Vorgang um an die zu indexierenden Daten zu kommen nur komplizierter gestaltet wird. Diese Funktion könnte allerdings für Datenbanken ohne direkten Zugriff interessant sein. 

Nachdem nun eine Suchmaschine ausgewählt wurde, ging es nun darum diese ordentlich aufzusetzen. Dabei wurde in dieser Arbeit Docker-Compose verwendet. Die Kommunikation zwischen den einzelnen virtuellen Containern wurde hierbei mit selbst generierten Zertifikaten verschlüsselt. Dabei kam es zu einigen Problemen mit der Generierung und Verwendung der Zertifikate, weshalb darüber nachgedacht werden sollte, ob die Verschlüsselung innerhalb des Systems zielführend ist. 

Im letzten Schritt wurde nun noch eine prototypische Implementierung in das Projekt vorgenommen. Dafür wurde ein Index mit allen für die Suche wichtigen Daten aufgebaut. !!Index verfeinern!!


Zudem wurde für einen Vergleich noch ein Index über alle Lemmata aufgebaut. Dieser ist der aktuell am langsamsten ladende Teil des Projekts. Mit dem Wechsel auf ElasticSearch ist es so gelungen die Laufzeit von diesem Query, um 50 \% zu verringern. Die Suche für die Nutzer wurde verbessert, indem nun mehr verschiedene Sucharten unterstützt werden. Auch ist es nun möglich mehr als 1001 Ergebnisse zu erhalten. Dies war vorher eine durch die Datenbank auferlegte Grenze. Um zu zeigen, was die Suchmaschine sonst noch für Funktionen unterstützt wurde zudem eine Funktion eingebaut, die die zehn Autoren auflistet, welche die meisten Artikel in der aktuellen Suche geschrieben haben. 

Zur Implementierung wurde der offizielle Klient von ElasticSearch verwendet, welcher auf einer sehr niedrigen Ebene arbeitet. Es gibt auch Klienten, welche das Level ein wenig mehr abstrahieren und so eine angenehmere Erfahrung bieten, allerdings diese alle nicht offiziell unterstützt. Daher habe ich mich in dieser Arbeit auf den eher simplen Klienten von ElasticSearch fokussiert. 

Sobald die Suchmaschine in das Projekt eingegliedert ist, können viele weitere Probleme des Projektes gelöst werden. So können zum Beispiel Synonymlisten für Autoren geführt werden, um die verschiedenen Schreibweisen bestimmter Autoren auszugleichen. Auch ist es mit der Suchmaschine möglich dem DDC-Baum, welcher schon seit langer Zeit implementiert werden sollte, leichter einzubauen. Zudem bietet ElasticSearch Funktionen zu Autokorrektur, welche die Sucherfahrung positiv bereichern können. Und für die Entwickler nimmt ElasticSearch einiges an Problemen mit der Datenbank ab. Aktuell werden viele Felder mithilfe von Triggern und Funktionen erstellt. Diese Trigger können nun auf Logstash übertragen werden, um so die Datenbank zu entlasten.
% ...
%--------------------------------------------------------------------------
\backmatter                        		% Anhang
%-------------------------------------------------------------------------
\bibliographystyle{IEEEtranHS.bst}			% Literaturverzeichnis
\bibliography{literatur}     			% BibTeX-File literatur.bib
%--------------------------------------------------------------------------
\printindex 							% Index (optional)
%--------------------------------------------------------------------------
\begin{appendix}						% Anhänge sind i.d.R. optional
   \chapter{Glossar}

\abbreviation{ESE}      {Enterprise Search Engine}
\abbreviation{Facetten} {Filter in Bibliothekarssprache}
			% Glossar   
   \chapter{Erklärung der Kandidatin / des Kandidaten}

\begin{description}[$\Box$~]
\item[$\Box$] Die Arbeit habe ich selbstständig verfasst und keine anderen als die angegebenen Quellen und Hilfsmittel verwendet.
\end{description}

\vspace{2cm}

\begin{minipage}[t]{3cm}
\rule{3cm}{0.5pt}
Datum
\end{minipage}
\hfill
\begin{minipage}[t]{9cm}
\rule{9cm}{0.5pt}
Unterschrift der Kandidatin / des Kandidaten
\end{minipage}	% Selbstständigkeitserklärung
\end{appendix}

\end{document}
