%%%%%%%%%%%%%%%%%%% vorlage.tex %%%%%%%%%%%%%%%%%%%%%%%%%%%%%
%
% LaTeX-Vorlage zur Erstellung von Projekt-Dokumentationen
% im Fachbereich Informatik der Hochschule Trier
%
% Basis: Vorlage svmono des Springer Verlags
%
%%%%%%%%%%%%%%%%%%%%%%%%%%%%%%%%%%%%%%%%%%%%%%%%%%%%%%%%%%%%%

\documentclass[envcountsame,envcountchap, deutsch]{i-studis}

\usepackage{makeidx}         	% Index
\usepackage{multicol}        	% Zweispaltiger Index
%\usepackage[bottom]{footmisc}	% Erzeugung von Fu�noten

%%-----------------------------------------------------
%\newif\ifpdf
%\ifx\pdfoutput\undefined
%\pdffalse
%\else
%\pdfoutput=1
%\pdftrue
%\fi
%%--------------------------------------------------------
%\ifpdf
\usepackage[pdftex]{graphicx}
\usepackage{epstopdf}
\usepackage[pdftex,plainpages=false]{hyperref}
%\else
%\usepackage{graphicx}
%\usepackage[plainpages=false]{hyperref}
%\fi

%%-----------------------------------------------------
\usepackage{color}				% Farbverwaltung
%\usepackage{ngerman} 			% Neue deutsche Rechtsschreibung
\usepackage[english, ngerman]{babel}
\usepackage[latin1]{inputenc} 	% Erm�glicht Umlaute-Darstellung
%\usepackage[utf8]{inputenc}  	% Erm�glicht Umlaute-Darstellung unter Linux (je nach verwendetem Format)

%-----------------------------------------------------
\usepackage{listings} 			% Code-Darstellung
\lstset
{
	basicstyle=\scriptsize, 	% print whole listing small
	keywordstyle=\color{blue}\bfseries,
								% underlined bold black keywords
	identifierstyle=, 			% nothing happens
	commentstyle=\color{red}, 	% white comments
	stringstyle=\ttfamily, 		% typewriter type for strings
	showstringspaces=false, 	% no special string spaces
	framexleftmargin=7mm, 
	tabsize=3,
	showtabs=false,
	frame=single, 
	rulesepcolor=\color{blue},
	numbers=left,
	linewidth=146mm,
	xleftmargin=8mm
}
\usepackage{textcomp} 			% Celsius-Darstellung
\usepackage{amssymb,amsfonts,amstext,amsmath}	% Mathematische Symbole
\usepackage[german, ruled, vlined]{algorithm2e}
\usepackage[a4paper]{geometry} % Andere Formatierung
\usepackage{bibgerm}
\usepackage{array}
\hyphenation{Ele-men-tar-ob-jek-te  ab-ge-tas-tet Aus-wer-tung House-holder-Matrix Le-ast-Squa-res-Al-go-ri-th-men} 		% Weitere Silbentrennung bei Bedarf angeben
\setlength{\textheight}{1.1\textheight}
\pagestyle{myheadings} 			% Erzeugt selbstdefinierte Kopfzeile
\makeindex 						% Index-Erstellung


%--------------------------------------------------------------------------
\begin{document}
%------------------------- Titelblatt -------------------------------------
\title{Titel der Arbeit}
\project{Ausarbeitung zur Vorlesung Wissenschaftliches Arbeiten}
%--------------------------------------------------------------------------
\supervisor{Titel Vorname Name} 		% Betreuer der Arbeit
\author{Bearbeiter 1: Vorname Nachname \\Bearbeiter 2: Vorname Nachname \\Bearbeiter 3: Vorname Nachname}							% Autor der Arbeit
\groupid{GruppenID}
\address{Ort,} 							% Im Zusammenhang mit dem Datum wird hinter dem Ort ein Komma angegeben
\submitdate{Abgabedatum} 				% Abgabedatum
%\begingroup
%  \renewcommand{\thepage}{title}
%  \mytitlepage
%  \newpage
%\endgroup
\begingroup
  \renewcommand{\thepage}{Titel}
  \mytitlepage
  \newpage
\endgroup
%--------------------------------------------------------------------------
\frontmatter 
%--------------------------------------------------------------------------
\kurzfassung

%% deutsch
\paragraph*{German}
Diese Arbeit handelt von der Analyse diverser Enterprise-Suchmaschinen für das DietrichOnline-Projekt \ref{dietrichonline}. Dabei wurden die Suchmaschinen nach einer Anforderungsliste untersucht und die verbleibenden Kandidaten für einen Ersteindruck aufgesetzt. 

Nachdem sich für Elasticsearch entschieden wurde, wurde diese in einer Docker-Umgebung aufgesetzt. Dabei wurde auf eine verschlüsselte Kommunikation zwischen den einzelnen Systemen viel Wert gelegt.

Im letzten Teil der Arbeit wurde zudem eine prototypische Implementierung in das Dietrich Online Projekt vorgenommen. Dafür wurde die Suche, sowie die Auto-Vervollständigung auf die Suchmaschine umgezogen.

%% englisch
\paragraph*{English}

This thesis analyzes various enterprise search-engines for the Dietrich Online project \ref{dietrichonline}. The search-engines were checked via a feature list and four of the remaining search engines were set up for a first impression.

After the decision was made for Elasticsearch, it was set up in a Docker environment. Great importance was attached to encrypted communication between the individual systems.

The last part of this thesis is a prototype implementation of the search engine in the Dietrich-Online project. The search and the auto-completion function were set up to use Elasticsearch. 			% Kurzfassung Deutsch/English
\tableofcontents 						% Inhaltsverzeichnis
%--------------------------------------------------------------------------
\mainmatter                        		% Hauptteil (ab hier arab. Seitenzahlen)
%--------------------------------------------------------------------------
% Die Kapitel werden in separaten .tex-Dateien abgelegt und hier eingebunden.
\chapter{Einleitung und Problemstellung}


Die Suche des DietrichOnline-Projektes \ref{dietrichonline} arbeitet aktuell auf einer MariaDB Datenbank. In dieser werden bei jeder Suchanfrage diverse Tabellen mithilfe von SQL-Joins zusammengebaut und daraufhin dem Nutzer ausgegeben. Bei den Datenmengen, welche sich aktuell in der Datenbank befinden, circa 1.4 Millionen Einträge, werden Ladezeiten unangenehm lang. Daher wurden die maximale Anzahl von Suchergebnissen, welche ein Nutzer aktuell bekommen kann, auf 1001 begrenzt. 

Damit die Nutzer ein möglichst gutes Sucherlebnis haben, sollen sogenannte Enterprise-Suchmaschinen evaluiert werden. Diese indexieren die Daten in einer Weise, welche es ermöglicht, viele Datensätze schnell zu durchsuchen. 

Im ersten Schritt werden nun diverse Suchmaschinen nach einer Kriterienliste analysiert. Im zweiten Schritt werden die vier am besten passenden Suchmaschinen daraufhin für einen Ersteindruck aufgesetzt.

Sobald ein Kandidat ausgewählt ist, wird dieser, wenn möglich in einer Docker-Umgebung aufgesetzt. Dabei werden auch die benötigten Datensätze indexiert.

Zuletzt wird eine prototypische Implementierung in das DietrichOnline-Projekt vorgenommen. Hierbei wird die aktuelle Suche durch die Enterprise-Suchmaschine ersetzt und um einige Funktionen, wie eine bessere Schnellsuche, erweitert. 
\chapter{Weitere Kapitel}

Die Gliederung hängt natürlich vom Thema und von der Lösungsstrategie ab. Als nützliche
Anhaltspunkte können die Entwicklungsstufen oder -schritte zum Beispiel der Softwareentwicklung betrachtet werden. Nützliche Gesichtspunkte erhält und erkennt man, wenn man sich
\begin{itemize}
  \item in die Rolle des Lesers oder
  \item in die Rolle des Entwicklers, der die Arbeit zum Beispiel fortsetzen, ergänzen oder pflegen soll,
\end{itemize}
versetzt. In der Regel wird vorausgesetzt, dass die Leser einen fachlichen Hintergrund haben, zum Beispiel Informatik studiert haben. Das heißt nur in besonderen, abgesprochenen Fällen schreibt man in populärer Sprache, sodass auch Nicht-Fachleute die Ausarbeitung prinzipiell lesen und verstehen können.

Die äußere Gestaltung der Ausarbeitung hinsichtlich Abschnittformate, Abbildungen, mathematische Formeln usw. wird in \hyperref[Stile]{Kapitel~\ref*{Stile}} kurz dargestellt.
\chapter{LaTeX-Bausteine}\label{Stile}

Der Text wird in bis zu drei Ebenen gegliedert:

\begin{enumerate}
  \item Kapitel ( \verb \chapter{Kapitel} ), \index{Kapitel}
  \item Unterkapitel  ( \verb \section{Abschnitt} ) und
  \item Unterunterkapitel  ( \verb \subsection{Unterabschnitte} ).
\end{enumerate}

\section{Abschnitt}\index{Abschnitt}
Text der Gliederungsebene 2.


\subsection{Unterabschnitt} \index{Unterabschnitt}
Text der Gliederungsebene 3.
Text Text Text Text Text Text Text Text Text Text Text Text Text Text Text
Beispiel f�r Quelltext\index{Quelltext} \\[2 ex]
\noindent
\begin{minipage}{1.0\textwidth} \small
\begin{lstlisting}
	Prozess 1:
	
	Acquire();
		a := 1;
	Release();
	...
	Acquire();
	if(b == 0)
	{					
		c := 3;
		d := a;
	}				
	Release();
\end{lstlisting}
\end{minipage}

\vspace{2cm}
\noindent
\begin{minipage}{1.0\textwidth} \small
\begin{lstlisting}
	Prozess 2:
	
	Acquire();
		b := 1;
	Release();
	...
	Acquire();
	if(a == 0)
	{					
		c := 5;
		d := b;
	}				
	Release();
\end{lstlisting}
\end{minipage}
\vskip 1em

Gr��ere Code-Fragmente sollten im Anhang eingef�gt werden.

\section{Abbildungen und Tabellen}

Abbildung\index{Abbildung} und Tabellen\index{Tabelle} werden zentriert eingef�gt. Grunds�tzlich sollen sie
erst dann erscheinen, nach dem sie im Text angesprochen wurden (siehe Abb. \ref{a1}). Abbildungen und Tabellen (siehe Tabelle \ref{t1}) k�nnen
im (flie�enden) Text (\verb here ), am Seitenanfang (\verb top ), am Seitenende
(\verb bottom ) oder auch gesammelt auf einer nachfolgenden Seite (\verb page )
oder auch ganz am Ende der Ausarbeitung erscheinen. Letzteres sollte man nur
dann w�hlen, wenn die Bilder g�nstig zusammen zu betrachten sind und die
Ausarbeitung nicht zu lang ist ($< 20$ Seiten).

\begin{figure} %[hbtp]
	\centering
		\includegraphics{images/p1ReadSeq.pdf}
	\caption{Bezeichnung der Abbildung}
	\label{a1}
\end{figure}

\begin{table} %[hbtp]
	\centering
		\begin{tabular}{l | l l l l}
		\textbf{Prozesse} & \textbf{Zeit} $\rightarrow$ \\
		\hline
			$P_{1}$ & $W(x)1$ \\
			$P_{2}$ & & $W(x)2$ \\
			$P_{3}$ & & $R(x)2$ & & $R(x)1$\\
			$P_{4}$ & & & $R(x)2$ & $R(x)1$\\
		\end{tabular}
	\caption{Bezeichnung der Tabelle}
	\label{t1}
\end{table}


\section{Mathematische Formel}\index{Formel}
Mathematische Formeln bzw. Formulierungen k�nnen sowohl im
laufenden Text (z.B. $y=x^2$) oder abgesetzt und zentriert im Text
erscheinen. Gleichungen sollten f�r Referenzierungen nummeriert
werden (siehe Formel \ref{gl-1}).
\begin{equation}
\label{gl-1}
e_{i}=\sum _{i=1}^{n}w_{i}x_{i}
\end{equation}

Entscheidungsformel:

\begin{equation}
\psi(t)=\left\{\begin{array}{ccc}
1 &  \qquad 0 <= t < \frac{1}{2} \\
-1 &  \qquad \frac{1}{2} <= t <1 \\
0 & \qquad sonst
\end{array} \right.
\end{equation}


Matrix:\index{Matrix}
\begin{equation}
A = \left(
\begin{array}{llll}
a_{11} & a_{12} & \ldots & a_{1n} \\
a_{21} & a_{22} & \ldots & a_{2n} \\
\vdots & \vdots & \ddots & \vdots \\
a_{n1} & a_{n2} & \ldots & a_{nn} \\
\end{array}
\right)
\end{equation}

Vektor:\index{Vektor} 

\begin{equation}
\overline{a} = \left(
\begin{array}{c}
a_{1}\\
a_{2}\\
\vdots\\
a_{n}\\
\end{array}
\right)
\end{equation}

\section{S�tze, Lemmas und Definitionen}\index{Satz}\index{Lemma}\index{Definition}

S�tze, Lemmas, Definitionen, Beweise,\index{Beweis} Beispiele\index{Beispiel} k�nnen in speziell daf�r vorgesehenen Umgebungen erstellt werden.

\begin{definition}(Optimierungsproblem)

Ein \emph{Optimierungsproblem} $\mathcal{P}$ ist festgelegt durch ein Tupel
$(I_\mathcal{P}, sol_\mathcal{P}, m_\mathcal{P}, goal)$ wobei gilt

\begin{enumerate}
\item $I_\mathcal{P}$ ist die Menge der Instanzen,
\item $sol_\mathcal{P} : I_\mathcal{P} \longmapsto \mathbb{P}(S_\mathcal{P})$ ist eine Funktion, die jeder Instanz $x \in I_\mathcal{P}$ eine Menge zul�ssiger L�sungen zuweist,
\item $m_\mathcal{P} : I_\mathcal{P} \times S_\mathcal{P} \longmapsto \mathbb{N}$ ist eine Funktion, die jedem Paar $(x,y(x))$ mit $x \in I_\mathcal{P}$ und $y(x) \in sol_\mathcal{P}(x)$ eine
Zahl $m_\mathcal{P}(x,y(x)) \in \mathbb{N}$ zuordnet (= Ma� f�r die L�sung $y(x)$ der Instanz $x$), und
\item $goal \in \{min,max\}$.
\end{enumerate}

\end{definition}

\begin{example} MINIMUM TRAVELING SALESMAN (MIN-TSP)
\begin{itemize}
\item $I_{MIN-TSP} =_{def}$ s.o., ebenso $S_{MIN-TSP}$
\item $sol_{MIN-TSP}(m,D) =_{def} S_{MIN-TSP} \cap \mathbb{N}^m$ 
\item $m_{MIN-TSP}((m,D),(c_1, \ldots , c_m)) =_{def} \sum_{i=1}^{m-1} D(c_i, c_{i+1}) + D(c_m,c_1)$ 
\item $goal_{MIN-TSP} =_{def} min$
\end{itemize}
\begin{flushright}
$\qed$
\end{flushright}
\end{example}

\begin{theorem} Sei $\mathcal{P}$ ein \textbf{NP}-hartes Optimierungsproblem.
Wenn $\mathcal{P} \in$ \textbf{PO}, dann ist \textbf{P} = \textbf{NP}.
\end{theorem}

\begin{proof} Um zu zeigen, dass \textbf{P} = \textbf{NP} gilt, gen�gt es
wegen Satz A.30 zu zeigen, dass ein einziges \textbf{NP}-vollst�ndiges
Problem in \textbf{P} liegt. Sei also $\mathcal{P}'$ ein beliebiges \textbf{NP}-vollst�ndiges Problem.

Weil $\mathcal{P}$ nach Voraussetzung \textbf{NP}-hart ist, gilt insbesondere
$\mathcal{P}' \leq_T \mathcal{P}_C$. Sei $R$ der zugeh�rige
Polynomialzeit-Algorithmus dieser Turing-Reduktion.
Weiter ist $\mathcal{P} \in$ \textbf{PO} vorausgesetzt, etwa verm�ge eines
Polynomialzeit-Algorithmus $A$. Aus den beiden
Polynomialzeit-Algorithmen $R$ und $A$ erh�lt man nun
leicht einen effizienten Algorithmus f�r $\mathcal{P}'$: Ersetzt man
in $R$ das Orakel durch $A$, ergibt dies insgesamt eine polynomielle
Laufzeit. 
%\begin{flushright}
$\qed$
% \end{flushright}
\end{proof}

\begin{lemma} Aus \textbf{PO} $=$ \textbf{NPO} folgt \textbf{P} $=$ \textbf{NP}.
\end{lemma}

\begin{proof} Es gen�gt zu zeigen, dass unter der angegeben
Voraussetzung KNAPSACK $\in$ \textbf{P} ist.

Nach Voraussetung ist MAXIMUM KNAPSACK $\in$ \textbf{PO},
d.h. die Berechnung von $m^*(x)$ f�r jede Instanz $x$ ist
in Polynomialzeit m�glich. Um KNAPSACK bei Eingabe
$(x,k)$ zu entscheiden, m�ssen wir nur noch $m^*(x) \geq k$
pr�fen. Ist das der Fall, geben wir $1$, sonst $0$ aus. Dies
bleibt insgesamt ein Polynomialzeit-Algorithmus. 
\begin{flushright}
$\qed$
\end{flushright}
\end{proof}

\section{Fu�noten}

In einer Fu�note k�nnen erg�nzende Informationen\footnote{Informationen die f�r die Arbeit zweitrangig sind, jedoch f�r den Leser interessant sein k�nnten.} angegeben werden. Au�erdem kann eine Fu�note auch Links enthalten. Wird in der Arbeit eine Software (zum Beispiel Java-API\footnote{\url{http://java.sun.com/}}) eingesetzt, so kann die Quelle, die diese Software zur Verf�gung stellt in der Fu�note angegeben werden.

\section{Literaturverweise}\index{Literatur}
Alle benutzte Literatur wird im Literaturverzeichnis angegeben\footnote{Dazu wird ein sogennanter bib-File, literatur.bib verwendet.}. Alle angegebene Literatur sollte mindestens einmal im Text referenziert werden\cite{Coulouris:02}.
\input{chapters/Beispiel}
\chapter{Zusammenfassung und Ausblick}

Diese Bachelorarbeit hat sich ausführlich mit Enterprise Suchmaschinen auseinandergesetzt, diese Verglichen und letztendlich eine in das Dietrich Online Projekt implementiert. 

Im ersten Schritt wurden diverse Suchmaschinen erstmal nach einer Anforderungsliste verglichen. Dafür wurde eine Tabelle erstellt, welche alle Suchmaschinen anhand der gefundenen Funktionen verglichen. Mithilfe dieser Basis wurden vier Suchmaschinen für den genaueren Vergleich herausgesucht.

Für den genaueren Vergleich wurden diese Suchmaschinen nacheinander aufgesetzt und einige Dokumente indexiert. Dabei musste die Suchmaschine selbständig die Daten aus der Suchmaschine laden und indexieren. Zudem wurde auch die Benutzerfreundlichkeit untersucht. Dafür wurde die Oberfläche, insofern eine vorhanden war, und die Dokumentation bewertet. Zum Schluss wurde daraufhin eine Suchmaschine ausgewählt, welche in das Dietrich Online Projekt implementiert werden sollte. Dabei war es aufgrund der Zeit leider nicht möglich einen korrekten wissenschaftlichen Vergleich zu erstellen. Es wurde lediglich ein Ersteindruck gewonnen.

Als Nächstes wurde über die Möglichkeit nachgedacht einen OAI Harverester vor die Datenbank zu stellen, um eine normierte Schnittstelle zwischen der Datenbank und Suchmaschine herzustellen. Nach einer kurzen Analyse wurde diese Methodik allerdings verworfen, da ein direkter Zugriff auf die Datenbank möglich ist und somit der Vorgang um an die zu indexierenden Daten zu kommen nur komplizierter gestaltet wird. Diese Funktion könnte allerdings für Datenbanken ohne direkten Zugriff interessant sein. 

Nachdem nun eine Suchmaschine ausgewählt wurde, ging es nun darum diese ordentlich aufzusetzen. Dabei wurde in dieser Arbeit Docker-Compose verwendet. Die Kommunikation zwischen den einzelnen virtuellen Containern wurde hierbei mit selbst generierten Zertifikaten verschlüsselt. Dabei kam es zu einigen Problemen mit der Generierung und Verwendung der Zertifikate, weshalb darüber nachgedacht werden sollte, ob die Verschlüsselung innerhalb des Systems zielführend ist. 

Im letzten Schritt wurde nun noch eine prototypische Implementierung in das Projekt vorgenommen. Dafür wurde ein Index mit allen für die Suche wichtigen Daten aufgebaut. !!Index verfeinern!!


Zudem wurde für einen Vergleich noch ein Index über alle Lemmata aufgebaut. Dieser ist der aktuell am langsamsten ladende Teil des Projekts. Mit dem Wechsel auf ElasticSearch ist es so gelungen die Laufzeit von diesem Query, um 50 \% zu verringern. Die Suche für die Nutzer wurde verbessert, indem nun mehr verschiedene Sucharten unterstützt werden. Auch ist es nun möglich mehr als 1001 Ergebnisse zu erhalten. Dies war vorher eine durch die Datenbank auferlegte Grenze. Um zu zeigen, was die Suchmaschine sonst noch für Funktionen unterstützt wurde zudem eine Funktion eingebaut, die die zehn Autoren auflistet, welche die meisten Artikel in der aktuellen Suche geschrieben haben. 

Zur Implementierung wurde der offizielle Klient von ElasticSearch verwendet, welcher auf einer sehr niedrigen Ebene arbeitet. Es gibt auch Klienten, welche das Level ein wenig mehr abstrahieren und so eine angenehmere Erfahrung bieten, allerdings diese alle nicht offiziell unterstützt. Daher habe ich mich in dieser Arbeit auf den eher simplen Klienten von ElasticSearch fokussiert. 

Sobald die Suchmaschine in das Projekt eingegliedert ist, können viele weitere Probleme des Projektes gelöst werden. So können zum Beispiel Synonymlisten für Autoren geführt werden, um die verschiedenen Schreibweisen bestimmter Autoren auszugleichen. Auch ist es mit der Suchmaschine möglich dem DDC-Baum, welcher schon seit langer Zeit implementiert werden sollte, leichter einzubauen. Zudem bietet ElasticSearch Funktionen zu Autokorrektur, welche die Sucherfahrung positiv bereichern können. Und für die Entwickler nimmt ElasticSearch einiges an Problemen mit der Datenbank ab. Aktuell werden viele Felder mithilfe von Triggern und Funktionen erstellt. Diese Trigger können nun auf Logstash übertragen werden, um so die Datenbank zu entlasten.
% ...
%--------------------------------------------------------------------------
\backmatter                        		% Anhang
%-------------------------------------------------------------------------
\bibliographystyle{geralpha}			% Literaturverzeichnis
\bibliography{literatur}     			% BibTeX-File literatur.bib
%--------------------------------------------------------------------------
\printindex 							% Index (optional)
%--------------------------------------------------------------------------
\begin{appendix}						% Anh�nge sind i.d.R. optional
   \chapter{Glossar}

\abbreviation{ESE}      {Enterprise Search Engine}
\abbreviation{Facetten} {Filter in Bibliothekarssprache}
			% Glossar   
\end{appendix}

\end{document}
