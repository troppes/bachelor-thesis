\chapter{Zusammenfassung und Ausblick}

Diese Bachelorarbeit hat sich ausführlich mit Enterprise-Suchmaschinen auseinandergesetzt, diese verglichen und letztendlich eine in das DietrichOnline-Projekt implementiert. Das Ziel dabei war es, eine geeignete Suchmaschine für dieses Projekt zu finden und implementieren.  

Im ersten Schritt wurden diverse Suchmaschinen mit einer Anforderungsliste verglichen. Dafür wurde eine Tabelle erstellt, welche alle Suchmaschinen anhand der gefundenen Funktionen verglichen. Mithilfe dieser Basis wurden vier Suchmaschinen für den genaueren Vergleich herausgesucht.

Für den genaueren Vergleich wurden diese Suchmaschinen nacheinander aufgesetzt und eine definierte Dokumentenanzahl indexiert. Dabei musste die Suchmaschine selbständig die Daten aus der Datenbank laden und indexieren. Zudem wurde auch die Benutzerfreundlichkeit untersucht. Dafür wurden die Oberfläche, insofern eine vorhanden war, und die Dokumentation bewertet. Zum Schluss wurde eine Suchmaschine ausgewählt, welche in das DietrichOnline-Projekt implementiert werden sollte. Dabei war es aufgrund der Zeit leider nicht möglich, einen tieferen Einblick in die Suchmaschinen zu liefern.

Als Nächstes wurde über die Möglichkeit nachgedacht einen OAI-Harvester vor die Datenbank zu stellen, um eine normierte Schnittstelle zwischen der Datenbank und Suchmaschine herzustellen. Nach einer kurzen Analyse wurde diese Methodik allerdings verworfen, da ein direkter Zugriff auf die Datenbank möglich ist und somit der Vorgang, um an die zu indexierenden Daten zu kommen, nur komplizierter gestaltet wird. Diese Funktion könnte allerdings für Datenbanken ohne direkten Zugriff interessant sein. 

Nach der Auswahl der Suchmaschine wurde diese auf einen Server installiert. Dabei wurde in dieser Arbeit Docker-Compose verwendet. Die Kommunikation zwischen den einzelnen virtuellen Containern wurde hierbei mit selbstgenerierten Zertifikaten abgesichert. Dabei kam es zu einigen Problemen bei der Generierung und Verwendung der Zertifikate, weshalb darüber nachgedacht werden sollte, ob die Verschlüsselung innerhalb des Systems zielführend ist, insofern das System weiterhin auf einem Server laufen soll. 

Im letzten Schritt wurde eine prototypische Implementierung in das Projekt vorgenommen. Dafür wurde ein Index mit allen für die Suche wichtigen Daten aufgebaut. Um die Größe des Indexes zu minimieren, wurde für alle Felder ein vorheriges Mapping vorgenommen. Zudem wurden extra Felder für eine Auto-Vervollständigungsfunktion indexiert. Mithilfe dieses Indexes wurde die Suche für die Nutzer verbessert. Es werden nun mehr verschiedene Sucharten unterstützt. Auch ist es möglich, mehr als 1001 Ergebnisse zu erhalten. Dies war vorher eine durch die Datenbank auferlegte Grenze. Um zu zeigen, was die Suchmaschine sonst noch für Funktionen unterstützt, wurde zudem eine Funktion eingebaut, die die zehn Autoren auflistet, welche die meisten Artikel in der aktuellen Suche geschrieben haben. 

Es wurde für einen Vergleich noch ein Index über alle Lemmata aufgebaut. Dieser ist der aktuell am langsamsten ladende Teil des Projekts. Mit dem Wechsel auf Elasticsearch ist es so gelungen, die Laufzeit dieser Abfrage um 50 \% zu verringern. 

Zur Implementierung wurde der offizielle Klient von Elasticsearch verwendet, welcher auf einer sehr niedrigen Ebene arbeitet. Es gibt auch Klienten, welche das Level ein wenig mehr abstrahieren und so eine einfachere Bedienung bieten, allerdings diese nicht offiziell unterstützt. Daher habe ich mich in dieser Arbeit auf den offiziellen Klienten fokussiert. 

Sobald die Suchmaschine in das Projekt eingegliedert ist, können viele weitere Probleme des Projektes gelöst werden. So können zum Beispiel Synonymlisten für Autoren geführt werden, um die verschiedenen Schreibweisen bestimmter Autoren auszugleichen. Auch ist es mit der Suchmaschine möglich das Feature DDC-Baum, welcher schon vor langer Zeit implementiert werden sollte, leichter umzusetzen. Zudem bietet Elasticsearch Funktionen zur Autokorrektur, welche die Sucherfahrung positiv bereichern kann. Und für die Entwickler nimmt Elasticsearch einiges an Problemen mit der Datenbank ab. Aktuell werden viele Felder mithilfe von Triggern und Funktionen erstellt. Diese Trigger können nun auf Logstash übertragen werden, um so die Datenbank zu entlasten.

Damit nicht bei jeder Anfrage eine Zertifikats-Autorität mit gereicht werden muss, kann auch noch ein sogenannter Reverse Proxy vor die Elasticsearch-Instanz gesetzt werden, welcher die Zertifikatsverwaltung erleichert.