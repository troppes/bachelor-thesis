\chapter{Genauer Vergleich}

In diesem Kapitel werden die vorher ausgewählten Suchmaschinen genauer verglichen. Dafür werden alle vier Suchmaschinen aufgesetzt und getestet. Dabei wird Wert auf alle Aspekte gesetzt. Wie leicht lässt sich die Suchmaschine aufsetzen? Wie ist, falls vorhanden, das Webinterface? Wie funktionieren die Query’s und vor Allem wie viel Zeit benötigen diese? Da ich dieses Projekt nicht nach dieser Bachelor-Arbeit wohl nicht weiter verfolgen kann, ist es auch wichtig zu schauen, wie leicht ein neuer Administrator sich in das System einlernen kann, beziehungsweise wie leicht das System zu verstehen und administrieren ist. Deshalb wird auch die Dokumentation verglichen und geschaut, wie groß die Community der einzelnen Suchmaschinen ist. 

\section{Aufbau der Tests}

Im ersten Schritt wird die Installation bewertet, dabei wird geschaut, wie einfach es ist die Software zu installieren. Hierbei ist es wichtig zu schauen, ob Grundbedingen gegeben sind, wie bestimmte andere Software, oder ob die Software keine Voraussetzungen hat, beziehungsweise andere Software mitliefert. Auch wird darauf geachtet, ob es einen Installations-Wizard zur Konfiguration gibt, oder ob alles Manuell konfiguriert werden muss.

Als Nächstes folgt der Ersteindruck der Software und des Interfaces. Dabei wird geschaut, wie übersichtlich die Oberfläche ist, falls eine gegeben ist, und wie verständlich das System für Einsteiger ist. Dafür wird im ersten Schritt möglichst auf die Dokumentation verzichtet, um einen Ersteindruck zu liefern und die Übersichtlichkeit für sich selbst sprechen lässt. Auch wird hier geschaut, wie viel sich über die Weboberfläche regeln lässt und was alles über das Terminal geregelt werden muss.

Schritt drei ist es sich die Dokumentation anzuschauen. Hierbei wird das Augenmerk auf die Übersichtlichkeit und Verständlichkeit gelegt. Da in diesen Kurztest nicht alle Funktionen durchgetestet werden können, ist es leider auch nicht möglich zu schauen, ob alle Funktionen korrekt und ausführlich dokumentiert sind. Sollte allerdings schon von den Grundfunktionen eine schlechte oder fehlende Dokumentation auffallen wird dies natürlich erwähnt. 

Im letzten Schritt werden nun Query’s abgesetzt und geprüft, wie lange die Suchmaschinen für die Query’s brauchen. Dabei wird ein kurzes Script in PHP geschrieben, um auch den PHP-Connector oder die API zu überprüfen. Auch wir hier darauf geachtet, wie genau die Query’s gestellt werden müssen.

\section{Solr}

\subsection{Installation}

\subsection{Oberfläche}

\subsection{Dokumentation}

\subsection{Absetzen einer Anfrage und Integration in PHP}