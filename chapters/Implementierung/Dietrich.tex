\chapter{Dietrich}

In diesem Kapitel wird die Implementation ins Dietrich-Projekt genauer erläutert.

\section{Vorbereitung}

Zuerst muss der ElasticSearch Client in der Docker-Compose hinzugefügt werden. 

\subsection{ElasticSearch}

\begin{lstlisting}[language=XML, frame=single, label={lst:es01}] 
    POST /_security/api_key
    {
      "name": "dietrich-webiste",
      "role_descriptors": { 
        "role-a": {
          "cluster": ["all"],
          "index": [
            {
              "names": ["dietrich_*"],
              "privileges": ["read"]
            }
          ]
        }
      }
    }
\end{lstlisting}

EXPLAIN QUERY!
Optimierung von Wildcard zu Praefix
https://www.elastic.co/guide/en/elasticsearch/reference/current/query-dsl-prefix-query.html

RUBY erklären!