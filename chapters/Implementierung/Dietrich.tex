\chapter{Dietrich}

In diesem Kapitel wird die Implementation ins Dietrich-Projekt genauer erläutert.

\section{Vorbereitung}

Zuerst muss der ElasticSearch Client in der Docker-Compose hinzugefügt werden. 

\subsection{ElasticSearch}

\begin{lstlisting}[language=XML, frame=single, label={lst:es01}] 
	es01:
	image: docker.elastic.co/elasticsearch/elasticsearch:7.5.1
	container_name: es01
	environment:
		- "ES_JAVA_OPTS=-Xms4g -Xmx4g"
	ulimits:
		memlock: -1
	volumes:
		- /srv/elk/elasticSearch01/:/usr/share/elasticsearch/data
		- /srv/elk/config/elasticsearch.yml:
			/usr/share/elasticsearch/config/elasticsearch.yml
	ports:
		- 9200:9200
	networks:
		- elastic
\end{lstlisting}

