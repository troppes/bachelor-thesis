\chapter{Vergleich der Enterprise Search Engines}

In ersten Schritt werden diverse Enterprise Search Engines evaluiert. Dafür wurde eine Anforderungsliste mit den Mitarbeitern erstellt. Die Systeme welche bei diesem Vergleich nach Features verglichen. Die Suchmaschinen, die am besten abschneiden werden anschließend aufgesetzt und genauer verglichen.

Diese Liste zeigt alle Basis-Funktionen, die für die Bibliothek hier K.O. Kriterien sind.
\begin{itemize}
    \item Open Source oder Kostenlos
    \item Unterstützung von Facetten
    \item Ranking der Suchergebnisse
    \item Volltextsuche
    \item Support für PDF, SQL, XML
    \item Logging-Möglichkeit
\end{itemize}

Des Weiteren sind die folgenden Funktionen stark erwünscht, allerdings nicht ausschlaggebend zu Disqualifikation.

\begin{itemize}
    \item Support für PostgreSQL
    \item Backup Funktionen
    \item Auto-Korrektur und Auto-Vervollständigung
    \item Security Features
    \item PHP-Support
    \item bezahlter Support
\end{itemize}

Durch die begrenzten finanziellen Mittel und die lange Projektlaufzeit besteht die Notwendigkeit eine kostenfreie, im besten Fall sogar eine Open Source Suchmaschine zu finden. Auch äußerst wichtig ist der Support für Facetten, da Dietrich-Online als Suchmaschine den Nutzer einige Tools zum Verfeinern seiner Suchergebnisse zur Verfügung stellen will.
Da wir hier mit großen Datenmengen arbeiten ist das Ranking auch von größer Bedeutung. Es können nicht alle Daten gleichzeitig dargestellt werden, von daher sollten die besten Treffer auch zuerst angezeigt werden. Dabei ist die Transparenz sehr wichtig. Wie das Ranking funktioniert muss erklärt und auf der Webseite veröffentlicht werden.
Die Volltextsuche wird es möglich machen auch nach Schlüsselwörtern im Titel oder Beschreibungen zu suchen.
Der Support für die verschiedenen Dateiformaten ergibt sich dadurch, dass dieses Projekt stark gewachsen ist. Es gibt viele Prozessschritte, welche auf denselben Daten in verschiedenen Formen arbeiten. Darunter werden alle Einträge im XML Format bearbeitet, es gibt alle Scans als PDF und für die Webseite sind alle Daten nochmals in der Datenbank vorhanden. Und alle Daten sind wichtig, da die Mitarbeitern-Module auf anderen Daten arbeiten, als der Nutzer. Dabei ergibt sich allerdings auch das Problem, dass sich Daten in der Datenbank doppeln werden. Dies wird sicherlich später nochmals von Bedeutung.
Als letztes ist es noch wichtig, dass zumindest ein Fehler-Logging geboten wird, damit schnell und effizient Probleme mit dem System erkannt und gelöst werden können. Ein erweitertes Monitoring ist ein Bonuspunkt.

Ein Support für PostgreSQL ist für dieses Projekt nicht so wichtig, allerdings könnte es sein, dass der Server später auch andere Datenbanken verwaltet. Dadurch wäre ein Support für diese Datenbankstruktur wünschenswert.
Es gibt bei den Maschinen in der Bibliothek sowieso ein Backup-Maschinismus. Allerdings ist eine manuelle Backup-Lösung wünschenswert, um die Suchmaschine losgelöst zu sichern und gegebenenfalls auch einfach auf einen anderen Server umzuziehen zu können. 
Auto-Korrektur und Auto-Vervollständigung sind beide sehr interessant, um den Nutzer mehr Komfort-Funktionen bieten zu können, ohne selbst groß implementieren zu müssen. 
Die Sicherheitsfunktionen sind für die Suchmaschinen mit Web-Oberfläche interessant. Generell sollte der Server ja nur intern anzusprechen sein. Wenn es allerdings eine Web-Oberfläche gibt, kann es sein, dass diese per Reverse Proxy ansprechbar gemacht wird, um eine Administration aus dem Internet möglich zu machen. Daher wäre es gut, wenn der Server ein Login-System bietet.
Einen PHP-Connector, welcher Objekte zum Umgang mit der Suchmaschine bietet, wäre auch wünschenswert. Allerdings bieten einige Suchmaschinen auch die Möglichkeit über JSON Anfragen an die Suchmaschine zu stellen. Es sollte zumindest eine der beiden Methoden verfügbar sein, damit die Suchmaschine einfach von PHP aus zu erreichen ist.
Sollte es mal ein Problem geben, was nicht im Haus gelöst werden kann, wäre die Möglichkeit auf bezahlten Support von Vorteil.

\section{Apache Lucene Core}
\label{lucenecore}

Lucene Core ist eine Open Source Enterprise Search Engine von der Apache Foundation geschrieben in Java.

Das Lucene Projekt wurde im Jahre 1997 vom Entwickler Doug Cutting gestartet. 2001 ist es dann der Apache Foundation als Teil des Jakarta-Projekts beigetreten und wurde 2005 ein eigenes Hauptprojekt der Foundation. \cite{Wikipedia.2019c}

Lucene Core erfüllt alle der Grundanforderungen. Für das Monitoring gibt es eine Klasse, die es auch ermöglicht, dass langsame Query’s geloggt werden. Zudem bietet es Support für PostgreSQL und Auto-Korrektur/Auto-Vervollständigung. Da es keine Web-Oberfläche besitzt, gibt es auch keine weiteren Sicherheitsfunktionen. Einen PHP-Connector gibt es leider auch nicht, man müsste daher mit PHP direkte Systemaufrufe an Java machen. Bezahlten Support gibt es hier nicht, da dieses Projekt zur Apache Foundation gehört. \cite{TheApacheSoftwareFoundation.2019b}

\section{Terrier}
\label{terrier}

Terrier ist eine Open Source Enterprise Search Engine geschrieben in Java. Entwickelt und gepflegt wird diese von der University of Glasgow. Sie existiert bereit seit 10 Jahren und besitzt, laut Webseite, eine breite Nutzerbasis. 
Terrier erfüllt leider nicht alle Grundanforderungen, da es keine direkte Möglichkeit gibt SQL zu indexieren. Es gibt allerdings eine Möglichkeit das SQL in JSON zu konvertieren und dieses dann in die Suchmaschine einzupflegen. Auch scheint es keinen Support für Facetten gegeben.
\cite{McCreadie.2019}

\section{Sphinx}
\label{sphinx}

Sphinx ist eine Suchmaschine entwickelt von Andrew Aksyonoff. Das Akronym steht für „SQL Phrase Index“.\cite{SphinxTechnologiesInc.b} Bis zur Version 2 wurde sie aktiv Open Source entwickelt. Ab Version 3 wurde die Entwicklung Closed Source. Auf der Github-Seite steht: „The sources for 3.0 will also be posted here when we decide to make those publicly available.“ \cite{sphinxserach.2019}, also gibt es kein genaues Datum ob und wann die Version 3 Open Source geht. Version 3.1.1 wurde im Oktober 2018 veröffentlicht und seitdem lässt sich auch nichts mehr über das Projekt finden. Von daher ist davon auszugehen, dass das Projekt nicht mehr weitergeführt wird. 

Zu den Features ist festzuhalten, dass es keinen nativen PDF-Support in der Open-Source Version gibt, ab der Version 3 wurde jedoch ein Dokumenten-Speicher eingebaut. Allerdings werden die anderen Anforderungen alle erfüllt. Es existiert, laut Webseite, sogar ein bezahlter Support, allerdings ist fraglich, ob man mit der Firma noch in Kontakt treten kann. \cite{SphinxTechnologiesInc.2019}

\section{Apache Solr}
\label{solr}

Apache Solr ist eine, auf Lucene Core \ref{lucenecore} basierende, viel eingesetzte Search Engine von der Apache Foundation. Sie erweitert Lucene Core, um ein grafische Benutzeroberfläche und einige weitere Funktionen. 
Die Entwicklung dafür begann 2004 als ein internes Projekt von CNET um eine bessere Suche für die eigene Webseite zu bieten. Später im Jahre 2006 hat CNET dann den Source Code an die Apache Foundation weitergegeben. Dadurch wurde  es zu einem eigenen Projekt bei der Apache Foundation. Im Jahre 2009 wurde Solr dann in das Apache Lucene Projekt eingefügt. Dort wird es auch aktuell noch weiterentwickelt. \cite{Wikipedia.2019b}

Solr wird unter anderem von DuckDuckGo und Best Buy eingesetzt. Durch die Unterstützung von der Apache Foundation längerfristige Weiterentwicklung abzusehen. 

Da Solr zur Apache Foundation gehört, ist es Open Source. Es bietet viele Funktionen von Haus, womit es alle Grundanforderungen erfüllt und besitzt darüber hinaus auch Support für fast alle Bonus-Features. Einzig und allein gibt es keinen bezahlten Support, dafür allerdings eine große Community, welche man durch einen Mailing Listen oder IRC erreichen kann. \cite{TheApacheSoftwareFoundation.2019}

\section{ElasticSearch}
\label{elasticsearch}

Eine weitere große Enterprise Search Engine ist ElasticSearch. Auch dieses Projekt arbeitet auf der Basis von Lucene. Zu den bekanntesten Kunden zählen Ebay und Adobe. Gestartet wurde das Projekt in den jungen 2000ern von Shay Banon, um eine Verwaltung für die Rezepte seiner Frau zu schaffen. Im Juni 2012 haben sich dann Logstash, ein Logging Dienst, Kibana, ein UI für ElasticSearch, und ElasticSearch zusammengetan. So enstand der ELK-Stack. Die enstandene Firma nennt sich: ElasticSearch Incorporated. Seitdem wurden der Produktkatalog stetig erweitert und die Produkte weiterentwickelt. Viele der weiteren Produkte sind allerdings nicht mehr Open-Source oder kostenlos. Der ELK-Stack ist allerdings weiterhin, und es wurde versprochen, dass es so bleibt, kostenlos und ElasticSearch zudem auch als Open-Source Variante zu haben. Eine genauere Aussage, welche Features nur in der kostenlosen und nicht in der Open-Source Variante zu finden sind, finden sich in der Tabelle \ref{vglTable}.

ElasticSearch erfüllt alle der Grundanforderungen, auch in der Open-Source Variante. Auch viele der optionalen Features kann man in der Open Source Variante genießen. Einzig die Sicherheitsfunktionen, wie rollen-basierte Authentifizierung sind der kostenlosen Variante vorbehalten. Eine Möglichkeit auf bezahlten Support besteht auch. \cite{Elasticsearch.2019}

\section{Fess}
\label{fess}

Fess ist eine Enterprise Search Engine basierend auf ElasticSearch entwickelt von dem japanischen Unternehmen CodeLibs. Die Suchmaschine ist komplett Open-Source und wird unter der Apache-Lizenz entwickelt.

Die Suchmaschine erfüllt alle Grundanforderungen. Darüber hinaus bietet es Support für PostgreSQL, Backups (sogar über die Web-Oberfläche) und Auto-Korrektur und Vervollständigung. Es gibt keinen direkten PHP Support, allerdings können anfragen über JSON geschickt werden. Ein bezahlter Support ist auch über die Firma N2SM Incorporated. \cite{N2SM.2019} möglich. Bei dieser Arbeiten anscheinend auch einige der Entwickler von Fess. Sicherheitsfunktionen werden über rollen-basierte Authentifizierung mitgeliefert. \cite{CodeLibs.2019}

\section{Algolia}
\label{algolia}

Algolia ist eine cloud-basierte Search Engine, welche unter anderem von Twitch und Lacoste verwendet wird. Die Suchmaschine wird hierbei als SAAS (Software as a Service) angeboten.  Hierbei lädt man die Daten auf Algolia Server und dann daraufhin über eine API-Schnittstelle die Suchen auf den Daten in der Cloud ausführen.

Sie erfüllt alle Grundanforderungen, wobei allerdings in der kostenlosen Variante grade einmal 10 Tausend Einträge und 50 Tausend Operationen im Monat erlaubt sind. Auch die optionalen Anforderungen werden so weit alle erfüllt. Der bezahlte Support wird ab der Starter Edition für 30 Dollar im Monat mitgeliefert. \cite{Algolia.2019}

\section{Manticore Search}
\label{manticore}

Manticore Search Engine ist eine Open-Source Lösung basierend auf Sphinx \ref{sphinx}. Nachdem Sphinx Closed Source gegangen ist, wurde auf der letzten offenen Version die erste Version von Manticore Search entwickelt. Zu den großen Kunden zählen unter anderem Craigslist und Boardreader.

Manticore erfüllt fast alle Grundanforderungen, allerdings ist kein nativer PDF-Support gegeben. Es muss daher auf eine Konvertierung der Daten auf XML gesetzt werden. Es findet sich außerdem eine Unterstützung von PostgreSQL, sowie Auto-Korrektur und Vervollständigung. Auch werden Log-Dateien produziert. Zuletzt gibt es noch eine Option auf bezahlten Support. Die Supportkosten sind dabei direkt auf der Webseite angegeben und belaufen sich auf 3000 Dollar im Jahr für den Standard Support. \cite{ManticoreSoftwareLtd.2019}

\section{Xapian}
\label{xapian}

Xapian ist eine Open-Source Enterprise Suchmaschine, welche von Zeit-Online, der Universitätsbibliothek Köln und der Debian Webseite genutzt wird. Die Suchmaschine basiert auf Open Muscat, einer Suchmaschine, welche an der Cambridge Universität in den 1980ern von Dr. Martin Porter entwickelt wurde. In 2001, als Open Muscat Closed Source ging, haben sich einige Entwickler die letzte offene Version genommen und diese weiterentwickelt.

Sie erfüllt alle der Grundanforderungen, wenn auch Logging nur im Grundsinne erfüllt wird, da nur Fehlermeldungen ausgegeben werden. Des Weiteren bietet die Suchmaschine Support für PostgreSQL. Auch eine Replikations-Funktion wird mitgeliefert. Sie bietet auch Auto-Korrektur und Auto-Vervollständigung. Ein Login-System mit Sicherheitsfunktionen gibt es durch das Fehlende Frontend Administration nicht. Es gibt allerdings die Möglichkeit mit Omega eine CGI-Suche zu nutzen. Diese Suche bietet allerdings keine Administration, sondern nur eine grafische Oberfläche für Suchanfragen.

Auch gibt es eine Möglichkeit für bezahlten Support. Auf der Webseite werden zwei Firmen angegeben, welche bezahlten Support bieten. Allerdings funktioniert der Link aktuell nur für eine der beiden Firmen aktuell. Zudem ist ein PHP-Connector für die Suchmaschine vorhanden, was die Einbindung ist das Projekt vereinfacht. \cite{XAP.2019}

\section{Datafari}
\label{datafari}

Datafari ist eine Open-Source Enterprise Suchlösung vom französischen Entwickler France Labs. Das Entwicklerstudio wurde 2011 gegründet und hat sich es sich zum Ziel gemacht, die beste Open-Source Enterpise Suchlösung zu erstellen. \cite{Labs.2019} Als Fundament dafür wurde hierbei Solr verwendet. Dies wurde dann mit dem ELK-Stack für die Analyse gemischt. Zu den Kunden zählt unter anderem das Linux Magazin, welches diese Suchmaschine in einer ihrer Ausgaben vorstellt. \cite{MichaelBrandenburg.2019}

Die Suchmaschine erfüllt alle Grundanforderungen. Darüber hinaus bietet sie auch Support für PostgreSQL, Auto-Korrektur und Vervollständigung, sowie den bezahlten Support. Eine Backup-Funktion gehört zu den Premium-Funktionen, genauso wie erweiterte Sicherheitsfunktionen. Allerdings ist zumindest die Rollenbasierte-Authentifizierung auch in der Open-Source Variante zu haben. Einen direkt PHP-Connector gibt es nicht, allerdings wird eine HTTP-API zu verfügung gestellt, welche es ermöglicht per POST-Request Anfragen zu stellen. \cite{Labs.b}

\section {Tabellarischer Vergleich}

Alle Suchmaschinen die zumindest die Grundanforderungen erfüllen, werden hier in der Tabelle \ref{vglTable} nun nochmals aufgeführt für einen leichteren Vergleich.

\begin{table} %[hbtp]
	\centering
		\begin{tabular}{l | l | l | l | l | l | l | l | l}
		& \textbf{LC} & \textbf{SH} & \textbf{AS} & \textbf{ES}  & \textbf{FE} & \textbf{AG} & \textbf{XP} & \textbf{DF} \\
        \hline
        Open Source oder Kostenlos                  & x & x  & x & x  & x & x  & x & x \\
        Unterstützung von Facetten                  & x & x  & x & x  & x & x  & x & x \\
        Ranking der Suchergebnisse                  & x & x  & x & x  & x & x  & x & x \\
        Volltextsuche                               & x & x  & x & x  & x & x  & x & x \\
        Support für PDF, SQL, XML                   & x & x* & x & x  & x & x  & x & x \\
        Monitoring / Logging                        & x & x  & x & x  & x & x  & x? & x \\
        \hline
        Support für PostgreSQL                      & x & x  & x  & x  & x & x  & x & x \\
        Backup                                      & - & -  & x  & x  & x & x+ & - & - \\
        Auto-Korrektur und Vervollständigung        & x & x  & x  & x  & x & x  & x & x \\
        Security Features                           & - & -  & x- & x* & x & x  & - & x \\
        PHP Support                                 & - & x  & x  & x  & - & x  & x & - \\
        bezahlter Support                           & - & x  & -  & x  & x & x  & x & x \\
        \hline
        unter aktiver Entwicklung**                 & x & -  & x  & x  & x & x  & x & x \\
        offizielles Docker Image                    & - & -  & x  & x  & x & -  & - & x \\
        Synonym Support                             & x & x  & x  & x  & x & x  & x & x \\
        Web-Interface                               & - & -  & x  & x  & x & x  & - & x \\
        Plugin Support                              & - & x  & x  & x  & x & -  & - & - \\
        JSON oder RESTful API                       & - & x* & x  & x  & x & -  & x-- & x \\
        SQL-Like Query Support                      & - & x  & -  & x  & - & -  & - & - \\
		\end{tabular}
    \caption{Feature-Vergleich der verschiedenen Enterprise Suchmaschinen }
    \label{vglTable}

    *  = Feature nur in der kostenlosen Variante verfügbar. \\
    ** = Update innerhalb des letzten halben Jahres \\
    -- = Nur mit Omega CGI installiert \\
    +  = Anbieter kümmert sich um das Feature \\
    -  = Funktion nur per Plugin Implementiert \\

    Die Tabelle vergleicht einige Features der ausgewählten Search Engines. Dabei wurden die Namen aus Platzgründen wie folgt abgekürzt:

    \begin{itemize}
        \item LC = Lucene Core \ref{lucenecore}
        \item SH = Sphinx \ref{sphinx}
        \item AS = Apache Solr \ref{solr}
        \item ES = ElasticSearch \ref{elasticsearch}
        \item FE = Fess \ref{fess}
        \item AG = Algolia \ref{algolia}
        \item XP = Xapian \ref{xapian}
        \item DF = Datafari \ref{datafari}
    \end{itemize} 


\end{table}

\section{Vorauswahl}

Nach einem ersten Feature-Vergleich haben nur 7 Suchmaschinen die Grundanforderungen erfüllt. Davon werden nun 4 Stück in den genaueren Vergleich genommen, bei dem die Systeme nun aufsetze und teste. Ich gehe nun die Suchmaschinen der Reihe nach durch und gebe zur jeder eine Begründung warum oder warum sie es nicht in den genauen Vergleich geschafft hat.

\subsection{Lucene Core}

Lucene Core scheidet dadurch aus das es zum einen keine direkte Schnittstelle liefert, die gut mit PHP zur erreichen ist. Die einzige Möglichkeit wären direkte System-Calls, wodurch es schwerer ist, die Systeme voneinander zu separieren, zum Beispiel auf verschiedenen Server zu legen. Zum anderen gibt es für Lucene Core ja eine Erweiterung, namentlich Solr, welches alle diese Probleme löst. \cite{TheApacheSoftwareFoundation.2019b}

\subsection{Sphinx}

Sphinx wäre eine interessante Alternative gewesen, allerdings durch den Kommunikationsverlust und die gestoppten Updates (Es gab schon seit über einem Jahr kein Update mehr), ist dieses Projekt wohl als tot anzusehen. \cite{SphinxTechnologiesInc.2019}

\subsection{Solr}

Wie schon bei Lucene Core kurz angesprochen, liefert Solr viele der Funktionen, die in diesem Umfeld benötigt werden, direkt mit. Dazu besitzt es eine Web-Oberfläche zur Administration. Durch die aktive Entwicklung unter der Apache-Lizenz und die große Community ist auch eine Langzeit-Entwicklung sehr wahrscheinlich. Daher ist Solr die erste der vier Kandidaten für das genauere Testen. \cite{TheApacheSoftwareFoundation.2019}

\subsection{ElasticSearch}

Auch ElasticSearch basiert auf Lucene, ist aber im Gegensatz nicht komplett Open-Source und bietet auch eine kommerzielle Version an, was allerdings auch bedeutet, dass es bezahlten Support gibt. Die Community und der Kundenkreis ist auch groß, was eine Weiterentwicklung sehr wahrscheinlich macht. Auch diese Suchmaschine bietet eine Web-Oberfläche mit besonderem Augenmerk auf die Visualisierung der Daten, was für spätere Administratoren einen einfacheren Einstieg in die Administration liefern könnte. Daher wird auch ElasticSearch den genaueren Vergleich mit eingebunden. \cite{Elasticsearch.2019}

\subsection{Fess}

Fess ist eine Suchmaschine, welches auf ElasticSearch basiert, was ja seinerseits auf Lucene basiert. Von den Funktionen her bietet Fess, dank der Basis, viele Möglichkeiten. Es gibt auch kommerziellen Support, allerdings nur von einer japanischen Firma. Dadurch kann es schwere werden mit dem Support in Kontakt zu treten, was mich dazu veranlasst Datafari \ref{datafari} dieser Suchmaschine vorzuziehen. Von der Idee her machen die beiden Firmen ja etwas ziemlich Ähnliches. \cite{CodeLibs.2019}

\subsection{Algolia}

Als einzige Cloud-only Lösung im Vergleich, bietet Algolia eine interessante Alternative. Leider sind im kostenlosen Bereich nicht genügend Einträge speicherbar. Auch sind 50.000 Operationen zu wenig für die das Dietrich-Online Projekt. Von daher fällt diese Suchmaschine durch diese Limitationen raus. \cite{Algolia.2019}

\subsection{Xapian}

Xapian ist als einzige Suchmaschine ohne Web-Administration im engeren Vergleich. Durch die Nutzung der Suchmaschine für die Bibliothek Köln gibt es einen Kunden der Software, welcher einen ähnlichen Anwendungsfall besitzt. \cite{Xapian.2019} Dadurch und die Erfüllung vieler weiterer Kriterien kommt diese Suchmaschine auch in die engere Auswahl. \cite{XAP.2019}

\subsection{Datafari}

Datafari ist der letzte Kandidat, der es in die engere Auswahl schafft, wie oben schon erwähnt gewinnt diese Suchmaschine gegen Fess, durch die Entwicklung in Frankreich und der daher besser zu erreichende Support. Darüber hinaus ist es interessant zu sehen, ob das Entwicklerstudio schafft Solr sinnvoll zu erweitern und die Datenaufbereitung mit ElasticSearch so zu liefern, dass sich die Suchmaschine trotzdem noch wie aus einem Guss anfühlt. \cite{Labs.b}


\subsection{Abschluss}

Daraus ergibt sich nun, dass die folgenden Suchmaschinen es in die genauere Auswahl geschafft haben: die beiden Platzhirsche Apache Solr \ref{solr} und ElasticSearch \ref{elasticsearch}, sowie Datafari \ref{datafari} als Open-Source Erweiterung von Solr und Xapian \ref{xapian} als Lucene-freie Alternative. Ich hätte auch gerne eine Suchmaschine, welche auf Sphinx basiert dabei gehabt, allerdings, war die einzige noch aktive Alternative zu Sphinx direkt Manticore Search. Diese hat aber bisher noch keinen PDF-Import, welcher leider zwingend erforderlich ist. 


