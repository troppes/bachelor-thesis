\chapter{Vergleich der Enterprise Search Engines}

In diesem ersten Schritt werden sich diverse Enterprise Search Engine Systeme angeschaut und evaluiert. Dafür wurde eine Liste mit Anforderungen an das System erstellt. Die Anforderungen sehen wie folgt aus:

\begin{itemize}
    \item Open Source oder Kostenlos
    \item Unterstützung von Facetten
    \item Ranking der Suchergebnisse
    \item Volltextsuche
    \item Support für PDF, SQL, XML
    \item Logging-Möglichkeit
\end{itemize}

Des Weiteren sind folgende Features ein Bonus:

\begin{itemize}
    \item Support für PostgreSQL
    \item Backup
    \item statistische Auswertung
    \item Auto-Korrektur und Auto-Vervollständigung
    \item Login System mit Security
    \item bezahlter Support
\end{itemize}

\section{Apache Lucene Core}

Lucene Core ist eine Open Source Enterprise Search Engine von der Apache Foundation geschrieben in Java. Sie bildet die Basis für Solr, auf welches später noch eingegangen wird. Dabei bietet Lucene Core allerdings auch schon einige interessante Features, unter anderem eine Ranked Suche und Facettierung. \ref{vglTable}

Das Lucene Projekt wurde im Jahre 1999 vom Entwicker Doug Cutting gestartet. Es ist im September 2001 der Apache Foundation beigetreten, als ein der Jakarta Familie. 2005 wurde es dann zu einen eigenen Top-Level Projekt. 

\cite{TheApacheSoftwareFoundation.2019b}

\section{Terrier}


\section{Sphinx}
\label{sphinx}


\section{Apache Solr}

Apache Solr ist eine viel eingesetzte Search Engine von der Apache Foundation. Sie basiert auf Apache Lucene Core und erweitert dieses um ein Interface. Die Entwicklung dafür begann 2004 als ein internes Projekt von CNET. Im Jahre 2006 hat CNET den Source Code an die Apache Foundation weitergegeben und es wurde zu einem großen Part des Lucene Projektes. Es wird unter anderem von DuckDuckGo und Best Buy eingesetzt. Durch die Apache Foundation ist ein stabiler Langzeitpartner gegeben. 
Solr arbeitet auf einer REST-like API, welches es ermöglicht eine einfache Schnittstelle für das Dietrich Projekt bereitzustellen. Darüber hinaus bietet es Support für Facetten.
Als Bonus hat es noch Support für Suchvorschläge, was für das Dietrich Projekt auf von Interesse wäre.

Apache Solr hat keinen bezahlten Support, sondern einen kostenfreien Community Support.

\cite{TheApacheSoftwareFoundation.2019}

\section{ElasticSearch}

Eine weitere großes Enterprise Search Engine ist ElasticSearch. Auch dieses Projekt arbeitet auf der Basis von Lucene. Zwei der bekannten Kunden sind Adobe und Ebay. Auch ElasticSearch stellt eine REST-API bereit, allerdings gibt es auch Klienten für viele Programmiersprachen, darunter PHP. Dabei kann auch eine SQL-like Sprache verwendet werden, insofern auf Open-Source verzichtet wird und der Basic Plan genommen wird. \cite{ElasticSearchSub.2019}

Sie bietet den Vorteil, dass Support dazugebucht werden kann. Für eine grafische Oberfläche gibt es Kibana. Dies bietet nicht nur eine Oberfläche für Elastic Search, sondern lässt sich auch mit den Logstash einer Logging-Engine verbinden. Da aktuell auch noch nach einer Logging Lösung gesucht wird, ist dies ein nicht zu unterschätzender Pluspunkt. 

\cite{Elasticsearch.2019}

\section{Manticore Search}

Manticore Search Engine ist eine Open-Source Solution basierend auf Sphinx \ref{sphinx}. Nachdem Sphinx Closed-Source gegangen ist, wurde auf der letzten offenen Version die erste Version von Manticore Search entwickelt. Zu den großen Kunden zählen unter anderem Craigslist und Boardreader.

Manticore erfüllt fast alle Grund Anforderungen, allerdings ist kein nativer PDF-Support gegeben. Es muss daher auf eine Konvertierung der Daten auf XML gesetzt werden. Es findet sich außerdem eine Unterstützung von PostgreSQL, sowie Auto-Korrektur und Vervollständigung. Es gibt auch einen Query-Log. Zuletzt gibt es noch eine Option auf bezahlten Support. Die Supportkosten sind dabei direkt auf der Webseite angegeben und belaufen sich auf 3000 Dollar im Jahr für den Standard Support. 

\cite{ManticoreSoftwareLtd.2019}


\section{Xapian}

Xapian ist eine Open-Source Enterprise Suchmaschine, welche von Zeit-Online, der Universitätsbibliothek Köln und der Debian Webseite genutzt wird. Die Suchmaschine basiert auf Open Muscat, einer Suchmaschine, welche an der Cambridge Universität in den 1980ern von Dr. Martin Porter entwickelt wurde. In 2001, als Open Muscat Closed-Source ging, haben sich einige Entwickler die letzte offene Version geladen und diese weiterentwickelt.

Sie erfüllt alle der Grundanforderungen, wenn auch Logging nur im Grundsinne erfüllt wird, da nur Errors geschmissen werden. Des Weiteren bietet die Suchmaschine Support für PostgreSQL. Auch eine Replikations-Funktion wird mitgeliefert. Sie bietet auch Auto-Korrektur und Auto-Vervollständigung. Ein Login-System mit Sicherheitsfunktionen gibt es durch das Fehlende Frontend Administration nicht. Es gibt allerdings die Möglichkeit mit Omega eine CGI-Suche zu nutzen. Diese Suche bietet allerdings keine Administration, sondern nur eine grafische Oberfläche für Suchanfragen.

Auch gibt es eine Möglichkeit für bezahlten Support. Auf der Webseite werden zwei Firmen angegeben, welche bezahlten Support bieten. Allerdings funktioniert der Link aktuell nur für eine der beiden Firmen aktuell. Zudem ist ein PHP-Connector für die Suchmaschine vorhanden, was die Einbindung ist das Projekt vereinfacht.

\cite{XAP.2019}

\section {Vorauswahl}

\begin{table} %[hbtp]
	\centering
		\begin{tabular}{l | l | l | l | l | l}
		& \textbf{LC} & \textbf{AS} & \textbf{MC} & \textbf{ES}  & \textbf{XP} \\
        \hline
        Open Source                                 & x & x & x & x* & x \\
        Unterstützung von Facetten                  & x & x & x & x  & x \\
        Ranking der Suchergebnisse                  & x & x & x & x  & x \\
        statistische Auswertung (Metriken)          & - & x & x & x  & - \\
        Volltextsuche                               & x & x & x & x  & x \\
        Support für PDF, SQL, XML                   & x & x & x & x  & x \\
        Monitoring / Logging                        & - & x & x & x* & - \\
        Support für PostgreSQL                      & x & x & x & x  & x \\
        Backup                                      & - & x & - & x* & - \\
        Auto-Korrektur                              & x & x & x & x  & x \\
        Auto-Vervollständigung                      & x & x & x & x  & x \\
        Login System und Security                   & - & x & - & x* & - \\
        bezahlter Support                           & - & - & x & x  & - \\
        Wildcard Suche                              & x & x & x & x  & x \\
        Query Vorschläge                            & x & x & x & x  & x \\
        Synonym Support                             & x & x & x & x  & x \\
        PHP Support                                 & - & x & x & x  & x \\
        (Fast) Echtzeit Indizierung                 & x & x & x & x  & x \\
        Web-Interface                               & - & x & - & x  & - \\
        Plugin Support                              & x & x & x & x  & - \\
        JSON Interface                              & x & x & x & x  & - \\
        SQL-Like Query Support                      & x & - & x & x  & - \\
		\end{tabular}
	\caption{Feature-Vergleich der verschiedenen Enterprise Search Engines}
    \label{vglTable}

    Die Tabelle vergleicht einige Features der ausgewählten Search Engines. Dabei wurden die Namen aus Platzgründen wie folgt abgekürzt:

    \begin{itemize}
        \item LC = Lucene Core
        \item AS = Apache Solr
        \item MC = Manticore Search
        \item ES = Elastic Search
        \item XP = Xapian
    \end{itemize} 

    Der Stern bedeutet, dass ein Feature nur in der Kostenlosen und nicht der Open Source Variante verfügbar ist.

\end{table}


Nach einem ersten Überblick wurden nun Aufgrund der Auswahlkriterien diese Systeme zum genaueren Vergleich ausgewählt: Apache Solr, Manticore Search, ElasticSearch (in der kostenlosen Version) und Xapian. Lucene Core wird nicht genauer untersucht, da Solr ein umfassenderes Paket bietet, welches den gestellten Anforderungen mehr entspricht.