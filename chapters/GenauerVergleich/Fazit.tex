
\chapter{Fazit des Vergleiches}

Nachdem nun alle Systeme für einen Ersteindruck aufgesetzt worden sind, ist es nun an der Zeit, eine Suchmaschine auszuwählen. Dazu wurde ein Treffen mit einigen Mitarbeitern der Bibliothek einberufen, um die Ergebnisse zu diskutieren. 

Dabei wurde schnell Datafari durch die Probleme, welche beim Test auftraten, ausgeschlossen. Auch Xapian wurde durch die schwache Benutzerfreundlichkeit und der unfertigen Dokumentation abgelehnt. 

Verbleibend waren nun noch Solr und Elasticsearch. Entschieden wurde sich dann letztendlich für Elasticsearch. Die lag daran, dass Elasticsearch viele Sicherheitsfunktionen direkt mitliefert und wohl auch bald in der Universitätsbibliothek Köln verwendet wird. Zudem bietet Elasticsearch die einsteigerfreundliche Erfahrung des Testes. Daher wird nun im nächsten Kapitel mit Elasticsearch weitergearbeitet.