\chapter{Dietrich Online Projekt}
\label{dietrichonline}

Dietrich Online ist eine Datenbank, welche die urheberrechtlich frei gewordenen Dietrich Bände\footnote{Zeitschriften Index von Felix Friedrich Dietrich} durchsuchbar machen soll. 

Beim Start dieses Projektes wurden dafür alle Zeitschriften mit einer OCR eingelesen. Dabei kam es zu einigen Problemen mit der Qualität der Daten. Um diese Fehler auszugleichen werden alle Textdaten nochmals händisch durchsucht und dabei auch wie folgt erweitert.

Die Dietrich haben alle Lemmata mithilfe von Siglen\footnote{IDs} normiert. Jede Sigle verweist auf ein eigenes Lemma. Lemmata sind Schlagwörter, welche dann mit Zeitungen verknüpft werden. Um die Lemmata noch besser durchsuchbar zu machen, werden sie um Schlagwörter von der Dewey Decimal Classification (DDC)  \footnote{Klassifikation zur Ordnung von Wissen \cite{DeutscheNationalBibliothek.ddc}} und Gemeinsamen Normdatei (GND)  \footnote{Normdatei für Personen \cite{DeutscheNationalBibliothek.2019b}}
erweitert.

An diese Lemmata werden nun die passenden Zeitschriften gebunden. Um nun auch die Zeitschriften zu normieren werden die Titel der Zeitschriften an mithilfe der Zeitschriften-Datenbank (ZDB) ergänzt. 
\cite{UniversityofTrier.2016}