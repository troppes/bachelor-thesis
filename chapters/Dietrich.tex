\chapter{DietrichOnline-Projekt}
\label{dietrichonline}

DietrichOnline ist eine Datenbank, welche die Durchsuchung der urheberrechtlich frei gewordenen Dietrich Bände\footnote{Zeitschriften Index von Felix Friedrich Dietrich} ermöglichen soll.

Beim Start dieses Projektes wurden dafür alle Zeitschriften mit mithilfe einer OCR\footnote{OCR steht für Optical Character Recognition. Eine Methodik mit deren beispielsweise Bücher in bearbeitbarer Form in den Computer eingelesen werden.} eingelesen. Dabei kam es zu einigen Problemen mit der Qualität der Daten. Um diese Fehler auszugleichen, werden alle Textdaten nochmals händisch durchsucht und erweitert.

Die Dietrich-Bücher haben alle Lemmata\footnote{Lemmata sind Schlagwörter, welche dann mit Zeitungen verknüpft werden.} mithilfe von Siglen\footnote{Eine Art Id für die Lemmata} normiert. Jede Sigle verweist auf ein eigenes Lemma. Die Lemmata werden, um sie besser durchsuchbar zu machen, um Schlagwörter von der Dewey Decimal Classification (DDC)  \footnote{Klassifikation zur Ordnung von Wissen \cite{DeutscheNationalBibliothek.ddc}} und Gemeinsamen Normdatei (GND)  \footnote{Normdatei für Personen \cite{DeutscheNationalBibliothek.2019b}} ergänzt.

An diese Lemmata werden die passenden Zeitschriften gebunden. Um nun auch die Zeitschriften zu normieren, werden die Titel der Zeitschriften mithilfe der Zeitschriften-Datenbank (ZDB) erweitert. \cite{UniversityofTrier.2016}