\chapter{Einleitung und Problemstellung}


Die Suche des DietrichOnline-Projektes \ref{dietrichonline} arbeitet aktuell auf einer MariaDB Datenbank. In dieser werden bei jeder Suchanfrage diverse Tabellen mithilfe von SQL-Joins zusammengebaut und daraufhin dem Nutzer ausgegeben. Bei den Datenmengen, welche sich aktuell in der Datenbank befinden, circa 1.4 Millionen Einträge, werden Ladezeiten unangenehm lang. Daher wurden die maximale Anzahl von Suchergebnissen, welche ein Nutzer aktuell bekommen kann, auf 1001 begrenzt. 

Damit die Nutzer ein möglichst gutes Sucherlebnis haben, sollen sogenannte Enterprise-Suchmaschinen evaluiert werden. Diese indexieren die Daten in einer Weise, welche es ermöglicht, viele Datensätze schnell zu durchsuchen. 

Im ersten Schritt werden nun diverse Suchmaschinen nach einer Kriterienliste analysiert. Im zweiten Schritt werden die vier am besten passenden Suchmaschinen daraufhin für einen Ersteindruck aufgesetzt.

Sobald ein Kandidat ausgewählt ist, wird dieser, wenn möglich in einer Docker-Umgebung aufgesetzt. Dabei werden auch die benötigten Datensätze indexiert.

Zuletzt wird eine prototypische Implementierung in das DietrichOnline-Projekt vorgenommen. Hierbei wird die aktuelle Suche durch die Enterprise-Suchmaschine ersetzt und um einige Funktionen, wie eine bessere Schnellsuche, erweitert. 