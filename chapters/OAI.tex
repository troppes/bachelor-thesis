\chapter{Nutzung des Open Archives Initiative Protokolls für Metadaten}

Während des Projektes kam die Frage auf, ob das Open Archive Initiative Protokoll für die Datenerhaltung verwendet werden sollte. Daher wird dies im folgenden kurz geprüft.


\section{Open Archives Initiative Protokolls}

Das Open Archives Initiative Protocol for Metadata Harvesting (OAI-PMH) ist ein Protokoll zum Austausch von Metadaten. Dabei werden Daten per GET- oder POST-Methode angefragt. Als Antwort erhält man im Folgenden ein XML-Dokument. So können die Metadaten mit bestimmten Facetten abgefragt werden (zum Beispiel Autor). Dabei geht es allerdings darum, primär Änderungen weiterzugeben. So können durch dieses Protokoll neue Einträge oder Änderungen in der Datenbank weitergeben werden.
\cite{DeutscheNationalBibliothek.2019}

\section{OAI-Harvester}

Ein OAI-Harvester ist ein Programm, welches durchgehend einen Abgleich der Daten vollführt. Dabei lässt es sich die Änderungen mit einem List-Befehl von dem Server geben und gleicht diese danach mit der eigenen Struktur ab. Sollten dabei Unterschiede festgestellt werden, werden daraufhin die Änderungen auch beim Harvester eingefügt. So steht der Harvester immer mit dem Server auf einen Stand.
\cite{DeutscheNationalBibliothek.2019}

\section{Support der Enterprise-Suchmaschinen}

Bei den vorhin genannten Enterprise-Suchmaschinen gibt es keine mit nativen OAI-Harvester Support. Es gibt die Möglichkeit für manche der Suchmaschinen, ein solches Verhalten mithilfe von Plugins zu implementieren. Allerdings sind die meisten dieser Add-ons auch schon veraltet.

\section{Auswertung}

Durch eine fehlende Basisimplementierung des Protokolls in den einzelnen Suchmaschinen und der Möglichkeit eines direkten Zugriffs auf die Datenbank, sehe ich keinen Grund, dieses Protokoll zu verwenden. Es müsste ein Server vor die Datenbank installiert werden und ein Harvester vor der Elasticsearch-Instanz. Dies ist ein großer Mehraufwand, welcher bei diesem Anwendungsfall nicht notwendig ist. Sollte allerdings diese Suchmaschine ein übergreifendes System werden, kann darüber nachgedacht werden, die anderen Datenbanken per OAI-Harvester anzusprechen.